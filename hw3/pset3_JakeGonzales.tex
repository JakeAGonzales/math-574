%%%%%%%%%%%%%%%%%%%%%%%%%%%%%%%%%%%%%
% -*- LaTeX -*- %%%%%%%%%%%%%%%%%%%%%%%%%%%%%%%%%%%%%%%%%%%%%%%%%%%%%%
%
%%%%%%%%%%%%%%%%%%%%%%%%%%%%%%%%%%%%%%%%%%%%%%%%%%%%%%%%%%%%%%%%%%%%%%
%**start of header
\documentclass [10pt]{article}
\usepackage{epsfig}
\usepackage{amsmath,amsfonts,amsthm,amssymb}
\usepackage{setspace}
\usepackage{Tabbing}
\usepackage{fancyhdr}
\usepackage{lastpage}
\usepackage{extramarks}
\usepackage{chngpage}
\usepackage{graphicx,float,wrapfig}
\usepackage{amssymb}
\usepackage{mathtools}
\usepackage{esint}
\usepackage{mathrsfs}
\usepackage{cancel}
\usepackage{xcolor}  
% In case you need to adjust margins:
\topmargin=-0.45in %
\evensidemargin=0in %
\oddsidemargin=0in %
\textwidth=6.5in %
\textheight=9.0in %
\headsep=0.25in %
\newtheorem{theorem}{Theorem}[subsection]
\newtheorem{definition}[theorem]{Definition}
\newtheorem{claim}[theorem]{Claim}
\newtheorem{lemma}[theorem]{Lemma}
\newtheorem{example}[theorem]{Example}
\newtheorem{corollary}[theorem]{Corollary}
\newtheorem{proposition}[theorem]{Proposition}

\newcommand{\jg}[1]{{\color{blue} #1}}

\begin{document}
\begin{center}
{\bf Homework Problems}\\
Math 327, Winter 2025\\
Due 11:30 pm, January 29, 2025
\end{center}

\begin{center}
\jg{
    Jake Gonzales}
\end{center}

{\bf Instructions:} Please include the full problem statement in your submission.
All solutions must be written in legible handwriting
or typed (in each case, the text should be of a reasonable size). Your solutions to
all problems should be written in complete sentences, with proper grammatical
structure.
In all cases where external resources are consulted or used, proper citation must
be given. In addition,
you must provide information on collaboration with your submission: if you worked with others,
or consulted anyone aside from the course staff, in preparing the solutions, their
names should be
listed; if you didn't work with anyone, please indicate this.
If your solutions are not typed, you must scan your written solutions and submit
the digital copy. When submitting problems through LaTex, the LaTex source file
(.tex) must be included in the submission.


\jg{
\textbf{Collaborators: N/A }
}

\begin{enumerate}

\item (Ross 7.4) Give examples of

\jg{
Recall, a rational number $r$ can be expressed by $r = \frac{m}{n}$ where $n, m \in \mathbb{Z}$ and $n \neq 0$. An irrational number does not satisfy this property, and therefore its decimal representation will be non-repeating and non-terminating.  
}
\begin{enumerate}
\item A sequence $(x_n)_{n=1}^{\infty}$ of irrational numbers having a
limit that is a rational number. 

\jg{
Consider the sequence $(x_n)_{n=1}^{\infty}$ where $x_n = \frac{1}{e^n}$. The number $e$ is a well-known irrational number. For any positive integer $n$, $e^n$ is also irrational. Furthermore, the reciprocal of an irrational number is irrational, meaning that $\frac{1}{e^n}$ is irrational. 

Its clear to see that as $n$ grows larger, $e^n$ also grows very large, meaning that the sequence $\frac{1}{e^n}$ will go to 0, which is a rational number by definition. 
}

\item A sequence $(r_n)_{n=1}^{\infty}$ of rational numbers having a limit
that is an irrational number.

\jg{
Consider the sequence $(r_n)_{n=1}^{\infty}$ where $r_n = (1+\frac{1}{n})^n$. This was given as an example in the book as a sequence that converges to the irrational number $e$. We can observe for finite $n$, the sequence $r_n = (1+\frac{1}{n})^n$ is rational. This is because inside the parenthesis we have $(1+ \frac{1}{n})$ and since $n$ is a positive integer this is a rational number. Also the integer exponent of a rational number i.e. $r^n$ is the product of two rational numbers which is a rational number.

For some examples we look at $n=1$ which gives $(1+\frac{1}{1})^1 = 2, n=2$ gives $(1+\frac{1}{2})^2 = (\frac{3}{2})^2 = \frac{9}{4}, n=3$ gives $(1+\frac{1}{3})^3 = (\frac{4}{3})^3 = \frac{64}{27}$, etc. 

Looking at example 1 from section 7 of the book the approximate this sequence to $r_{1000} = 2.7169$ and explain that this sequence does in fact converge to $e$. 
}
\end{enumerate}
\clearpage

\item (Ross 8.1) Prove the following:
\begin{enumerate}
\item $\lim_{n \to \infty} \frac{(-1)^n}{n}=0$

\jg{
\textbf{\underline{Discussion.}}

Following definition 7.1 from the textbook, for some arbitrary $\epsilon > 0$, we want to show $| \frac{(-1)^n}{n} - 0 | < \epsilon$. Thus we have $| \frac{(-1)^n}{n}| < \epsilon$. We can rewrite this as $|(-1)^n| \cdot |\frac{1}{n}|$. Notice $|(-1)^n| = 1$ for any integer $n$, then we can write $1 \cdot |\frac{1}{n}|$. Hence, since $n \geq 0$, we want to show that $\frac{1}{n} < \epsilon$. By multiplying both sides by $n$ and dividing both sides $\epsilon$, we get $\frac{1}{\epsilon} < n$. 

Thus we see $n > \frac{1}{\epsilon}$ implies $|\frac{(-1)^n}{n} - 0 | < \epsilon$. Suggesting that we should set $N = \frac{1}{\epsilon}$. \\

\textbf{\underline{Formal Proof.}}

Let $\epsilon > 0$. Let $N = \frac{1}{\epsilon}$. Then $n > N$ implies $|\frac{(-1)^n}{n} - 0 | < \epsilon = \frac{1}{n} < \epsilon$. \\
}
\item $\lim_{n \to \infty} \frac{1}{n^{1/3}}=0$

\jg{
\textbf{\underline{Discussion.}}

For some arbitrary $\epsilon > 0$, we want to show that $|\frac{1}{n^{1/3}} - 0| < \epsilon$. Thus we have $\frac{1}{n^{1/3}} < \epsilon$. Multiplying by ${n^{1/3}}$ and dividing by $\epsilon$ gives $\frac{1}{\epsilon} < {n^{1/3}}$. Finally, cubing both sides gives $\frac{1}{\epsilon^3} < n$. Thus we see $n > 1/\epsilon^3$ implies $|\frac{1}{n^{1/3}} - 0| < \epsilon$. Suggesting we set $N = 1/\epsilon^3$. \\

\textbf{\underline{Formal Proof.}}

Let $\epsilon > 0$. Let $N = 1/\epsilon^3$. Then $n > N$ implies $|\frac{1}{n^{1/3}} - 0| < \epsilon$. \\
}
\item $\lim_{n \to \infty} \frac{2n-1}{3n+2}=\frac{2}{3}$

\jg{
\textbf{\underline{Discussion.}}

For some arbitrary $\epsilon > 0$, we want to show that $|\frac{2n-1}{3n+2}=\frac{2}{3}| < \epsilon$. To subtract the inside of the absolute value we can cross multiply to find a common denominator, so that we get $|\frac{3(2n-1) - 2(3n+2)}{3(3n+2)}| < \epsilon$. Simplifying gives $|\frac{-7}{3(3n+2)}| < \epsilon$ or after applying the absolute values $|\frac{7}{3(3n+2)}| < \epsilon$. Multiplying by $(3n+2)$ and dividing by $\epsilon$ gives $\frac{7}{3 \epsilon} < 3n + 2$. So finally we have $\frac{7}{9 \epsilon} - \frac{2}{3} < n$. Thus we see $n > \frac{7}{9 \epsilon} - \frac{2}{3}$ implies $|\frac{2n-1}{3n+2}=\frac{2}{3}| < \epsilon$, suggesting we set $N = \frac{7}{9 \epsilon} - \frac{2}{3}$. \\

\textbf{\underline{Formal Proof.}}

Let $\epsilon > 0$. Let $N = \frac{7}{9 \epsilon} - \frac{2}{3}$. Then $n > N$ implies $|\frac{2n-1}{3n+2}-\frac{2}{3}| < \epsilon$. \\
}
\item $\lim_{n \to \infty} \frac{n+6}{n^2-6}=0$

\jg{
\textbf{\underline{Discussion.}}

For some arbitrary $\epsilon > 0$, we want to show that $|\frac{n+6}{n^2-6} - 0| < \epsilon$. Thus we have $|\frac{n+6}{n^2-6}| < \epsilon$. Since for $n > \sqrt{6}$ we have $(n^2 - 6)>0$, we can drop the absolute values, thus we have $\frac{n+6}{n^2-6} < \epsilon$. Here, it is difficult to isolate $n$. Recall, we need to find some $N$ such that $n>N$ implies $|\frac{n+6}{n^2-6}| < \epsilon$, but we do not need to find the least such $N$. We will do this by finding an upper bound for $\frac{n+6}{n^2 - 6}$. For $n>2$, we have $n^2 - 6 > \frac{n^2}{2}$, so 
\begin{align*}
    \frac{n+6}{n^2-6} < \frac{n+6}{\frac{n^2}{2}} = \frac{2(n+6)}{n^2}, 
\end{align*}
which by further simplifying gives $\frac{2(n+6)}{n^2} = \frac{2}{n} + \frac{12}{n^2}$. For $n>1$, $\frac{12}{n} > \frac{12}{n^2}$, so 
\begin{align*}
    \frac{2}{n} + \frac{12}{n^2} < \frac{2}{n} + \frac{12}{n} = \frac{14}{n}. 
\end{align*}
Thus we have $\frac{n+6}{n^2 - 6} < \frac{14}{n}$, hence we want $\frac{14}{n} < \epsilon$, implying $n > \frac{14}{\epsilon}$. Therefore, we choose $N = \max (\sqrt{6}, \frac{14}{\epsilon})$.

\textbf{\underline{Formal Proof.}}
Let $\epsilon>0$. Let $N = \max (\sqrt{6}, \frac{14}{\epsilon})$. Then for $n > N$, we have $|\frac{n+6}{n^2-6}-0| < \epsilon$.
}

\end{enumerate}
\clearpage

\item (Ross 8.2) Determine the limits of the following sequences, and then prove
your claims.
\begin{enumerate}
\item $a_n = \frac{n}{n^2+1}$

\jg{
\textbf{\underline{Discussion.}}

Suppose we rewrite the sequence $a_n = \frac{n}{n^2+1}$ as $a_n = \frac{1}{n + \frac{1}{n}}$. Then, as $n$ grows large, $\frac{1}{n}$ is very small so it is reasonable to conclude that $\lim a_n = 0$. Also, this is mentioned as a valid method in Example 2 on page 35 of the textbook. 

Let us now prove that $\lim a_n = 0$. Following a similar approach from problem 2, we want to show for an arbitrary $\epsilon > 0$ that $|\frac{n}{n^2 + 1} - 0| < \epsilon$ for large enough $n$. Or that $\frac{n}{n^2 + 1} < \epsilon$. Note that for $n \geq 1$, $n^2 + 1 > n^2$, so 
\begin{align*}
    \frac{n}{n^2 + 1} < \frac{n}{n^2} = \frac{1}{n}.
\end{align*}
Thus, it suffices to show that $\frac{1}{n} < \epsilon$, which implies $n > \frac{1}{\epsilon}$. Therefore, we choose $N = \frac{1}{\epsilon}$. \\

\textbf{\underline{Formal Proof.}}

Let $\epsilon > 0$. Let $N = \frac{1}{\epsilon}$. Then for all $n>N$, we have $|\frac{n}{n^2+1} - 0| < \epsilon$.
}
\item $b_n = \frac{7n-19}{3n+7}$

\jg{
\textbf{\underline{Discussion.}}

We can rewrite the sequence $b_n$ as $\frac{7 - (19/n)}{3 + (7/n)}$, therefore we can conclude that $\lim b_n = \frac{7}{3}$. To subtract, we will find a common denominator through cross multiplying, so
\begin{align*}
    \left|\frac{7n - 19}{3n+7} - \frac{7}{3}\right| = \left|\frac{3(7n-19) - 7(3n+7)}{3(3n+7)}\right|.
\end{align*}
Simplifying, 
\begin{align*}
    \left|\frac{3(7n-19) - 7(3n+7)}{3(3n+7)}\right| = \left|\frac{21n - 57 - 21n - 49}{9n + 21}\right| = \left|\frac{-106}{9n+21}\right|. 
\end{align*}
Since $(9n+21) > 0$, we drop the absolute value to get
\begin{align*}
    \frac{106}{9n+21} < \epsilon.
\end{align*}
To isolate $n$ we can rearrange to get $106 < \epsilon (9n+21)$ or $106 < 9\epsilon n + 21\epsilon$, which implies $n > \frac{106 - 21\epsilon}{9\epsilon}$. Thus, we choose $N = \frac{106 - 21\epsilon}{9\epsilon}$. \\

\textbf{\underline{Formal Proof.}}

Let $\epsilon > 0$. Let $N = \frac{106 - 21\epsilon}{9\epsilon}$. Then $n > N$ implies $|\frac{7n-19}{3n+7} - \frac{7}{3}| < \epsilon$. 
 
}

\item $c_n = \frac{4n+3}{7n-5}$

\jg{
\textbf{\underline{Discussion.}}
We can rewrite the sequence $c_n$ as $\frac{4 + (3/n)}{7 - (5/n)}$, therefore we can conclude that $\lim c_n = \frac{4}{7}$. To prove this, we want to show that for an arbitrary $\epsilon > 0$, we have $|\frac{4n+3}{7n-5} - \frac{4}{7}| < \epsilon$. To subtract, we can find a common denominator through cross multiplying, then we simplify after to get
\begin{align*}
    \left|\frac{4n+3}{7n-5}-\frac{4}{7}\right| 
    &= \left|\frac{7(4n+3)-4(7n-5)}{7(7n-5)}\right| \\
    &= \left|\frac{28n+21-28n+20}{49n-35}\right| \\
    &= \left|\frac{41}{49n-35}\right|.
\end{align*}

Since $(49n-35) > 0$, we drop the absolute value to get
\begin{align*}
    \frac{41}{49n-35}<\epsilon.
\end{align*}
To isolate $n$, we can rearrange to get $41 < \epsilon(49n-35)$ or $41 < 49\epsilon n-35\epsilon$, which implies $n > \frac{41+35\epsilon}{49\epsilon}$. Thus, we can choose $N = \frac{41+35\epsilon}{49\epsilon}$. \\

\textbf{\underline{Formal Proof.}}

Let $\epsilon > 0$. Let $N = \frac{41+35\epsilon}{49\epsilon}$. Then $n > N$ implies $|\frac{4n+3}{7n-5} - \frac{4}{7}| < \epsilon$. 
}
\item $d_n = \frac{2n+4}{5n+2}$

\jg{
\textbf{\underline{Discussion.}}

We can rewrite the sequence $d_n$ as $\frac{2 + (4/n)}{5 + (2/n)}$, therefore we can conclude that $\lim d_n = \frac{2}{5}$. To prove this, we want to show that for an arbitrary $\epsilon > 0$, we have $|\frac{2n+4}{5n+2} - \frac{2}{5}| < \epsilon$. To subtract, we can find a common denominator through cross multiplying, then we simplify after to get
\begin{align*}
    \left|\frac{2n+4}{5n+2}-\frac{2}{5}\right| 
    &= \left|\frac{5(2n+4)-2(5n+2)}{5(5n+2)}\right| \\
    &= \left|\frac{10n+20-10n-4}{25n+10}\right| \\
    &= \left|\frac{16}{25n+10}\right|.
\end{align*}
Since $(25n+10) > 0$, we drop the absolute value to get
\begin{align*}
    \frac{16}{25n+10}<\epsilon.
\end{align*}
To isolate $n$, we can rearrange to get $16 < \epsilon(25n+10)$ or $16 < 25\epsilon n+10\epsilon$, which implies $n > \frac{16-10\epsilon}{25\epsilon}$. Thus, we can choose $N = \frac{16-10\epsilon}{25\epsilon}$. \\

\textbf{\underline{Formal Proof.}}

Let $\epsilon > 0$. Let $N = \frac{16-10\epsilon}{25\epsilon}$. Then $n > N$ implies $|\frac{2n+4}{5n+2} - \frac{2}{5}| < \epsilon$.
}
\item $s_n = \frac{1}{n}\sin n$

\jg{
\textbf{\underline{Discussion.}}

The limit can be looked at in two parts. First lets look at the sequence $a_n = \frac{1}{n}$, which clearly as $n$ grows larger, the sequence $a_n$ goes to 0. Also, $b_n = \sin n$ is well known to oscillate in a bounded period between $1$ and $-1$. However, the sequence $b_n$ ensures that $s_n$ tends towards $0$ as $n$ grows larger. Also, by problem 4 of this homework, we know that this limit does in fact go to zero. 

We want to show that for any $\epsilon > 0$, we have $|\frac{1}{n} \sin n - 0| < \epsilon$ for large enough $n$. Since $|\sin n| \leq 1$, we have that $|\frac{1}{n} \sin n | \leq \frac{1}{n}$. Thus, $|\frac{1}{n} \sin n | < \epsilon$ whenever $\frac{1}{n} < \epsilon$, or $n > \frac{1}{\epsilon}$. Thus, choose $N = \frac{1}{\epsilon}$. \\

\textbf{\underline{Formal Proof.}}

Let $\epsilon > 0$. Let $N = \frac{1}{\epsilon}$. Then for $n > N$, we have $|\frac{1}{n}\sin n - 0| < \epsilon$. 

}
 
\end{enumerate}
\clearpage
\item (Ross 8.4) Let $(t_n)_{n=1}^{\infty}$ be a bounded sequence, and let $
(s_n)_{n=1}^{\infty}$ be a sequence such that $\lim_{n\to \infty} s_n= 0$. Prove
that $\lim_{n \to \infty} s_nt_n= 0$.

\jg{
\textbf{\underline{Discussion.}}

If the sequence $(t_n)$ is bounded then that means there exits $M > 0$ such that $|t_n| \leq M$ for all $n$. Since $\lim s_n = 0$, then for any $\epsilon > 0$, there exits an $N$ such that for all $n > N$, $|s_n| < \epsilon$. We want to show that $|s_n t_n|$ becomes small for large $n$, meaning $\lim s_n t_n = 0$. \\

\textbf{\underline{Formal Proof.}}

Since $(t_n)$ is bounded, there exists a constant $M > 0$ such that $|t_n| \leq M$ for all $n$. Since $\lim s_n = 0$, for any $\epsilon > 0$, there exists an $N$ such that for all $n > N$ we have $|s_n| < \frac{\epsilon}{M}$. Now consider $s_n t_n$. We can write $|s_n t_n| = |s_n| \cdot |t_n|$. Using the fact that $|t_n| \leq M$ and $|s_n| < \frac{\epsilon}{M}$, then 
\begin{align*}
    |s_n t_n| < M \cdot \frac{\epsilon}{M} = \epsilon. 
\end{align*}
Therefore, for any $\epsilon > 0$, there exits an $N$ such that for all $n > N, |s_n t_n| < \epsilon$. By definition, this means that $\lim s_n t_n = 0$. 
}

\clearpage
\item (Ross 8.5)
\begin{enumerate}
\item Consider three sequences $(a_n)_{n=1}^{\infty}$, $(b_n)_{n=1}^{\infty}$, and $(s_n)_{n=1}^{\infty}$ such that
\begin{align*}
a_n \leq s_n \leq b_n
\end{align*}
for all $n$ and $\lim_{n \to \infty} a_n = \lim_{n \to \infty} b_n = x$.
Prove that $\lim_{n \to \infty} s_n= x.$

\jg{
Let $\epsilon > 0$. We want to prove that $\lim s_n = x$ which means that there exists some $N$ that implies $|s_n - x| < \epsilon$ for all $n > N$. Expanding the absolute value means showing that $x - \epsilon < s_n < x + \epsilon$. We also know from the problem definition that $a_n \leq s_n \leq b_n$. Lets now look at the limit of those sequences. 

We know from the problem statement that $\lim a_n = x$ and $\lim b_n = x$. By definition, this means that there exists $N_1$ such that $|a_n - x| < \epsilon$ for $n > N_1$ and similarly that there exits $N_2$ such that $|b_n - x| < \epsilon$ for $n > N_2$. Expanding the absolute value gives us that $n > N_1$ implies $x - \epsilon < a_n$ and that $n > N_2$ implies $b_n < x + \epsilon$. 

Now, if we consider the $n$ such that $n > \max(N_1, N_2)$ then this implies $x - \epsilon < a_n \leq s_n \leq b_n < x + \epsilon$ and hence $|s_n - x| < \epsilon$. 
}

\item Suppose $(s_n)_{n=1}^{\infty}$ and $(t_n)_{n=1}^{\infty}$ are
sequences such that $|s_n|\leq t_n$ for all $n$ and $\lim_{n \to \infty} t_n=0$.
Prove $\lim_{n \to \infty} s_n =0.$

\jg{
We are given that $\lim_{n \to \infty} t_n=0$. We can conclude that $\lim_{n \to \infty} (-t_n)=0$, since multiplying by a constant $(-1)$ doesn't change the limit of the sequence. Then since we are given $|s_n| \leq t_n$ or $-t_n \leq s_n \leq t_n$, we can apply part (a) directly to show that $\lim s_n = 0$.  
}
\end{enumerate}
\clearpage

\item Prove that if $s_n = (-1)^n$, then $\lim_{n \to \infty} s_n$ does not exist.

\jg{
\textbf{\underline{Discussion.}}

We want to show that if $s_n  = (-1)^n$, then $\lim_{n \to \infty} s_n$ does not converge to some number $s$, i.e., the limit does not exist. Intuitively, we can see that the limit will alternate between the values $1$ and $-1$ for all integers $n \geq 0$. We want to show that there is no number $s$, even if its $1$ or $-1$, will serve as a limit for this sequence, since the sequence will always remain $1$ away from $s$. Specifically, this means showing that $|(-1)^n - s| < \epsilon$ will not hold for $\epsilon = 1$ and for all $n > N$. We can show this by contradiction.  \\

\textbf{\underline{Formal Proof.}}

Suppose, for the sake of contradiction, that $\lim_{n \to \infty} (-1)^n = s$ for some number $s$. Let $\epsilon = 1$. Therefore, there exists some $N$ such that 
\begin{align*}
    n > N \quad \text{implies} \quad |(-1)^n - s| < 1.
\end{align*}
Considering both an even and odd $n > N$, we get the following 
\begin{align*}
    |1-s| < 1 \quad \text{and} \quad |-1 - s | < 1. 
\end{align*}
We want to relate these two inequalities. Let's consider the distance between $1$ and $-1$ as $|1-(-1) = |2| = 2$. Using the triangle inequality, we can rewrite $|1-(-1)|$ as 
\begin{align*}
    |1-(-1)| = |1-s + s - (-1) \leq |1-s| + |s - (-1)|.
\end{align*}
Substituting the inequalities $|1-s| < 1$ and $|s - (-1)| < 1$ gives 
\begin{align*}
    2 = |1-(-1)| \leq |1-s| + |s-(-1)| < 1 + 1 = 2. 
\end{align*}
This is certainly not true. Therefore, our assumption that $\lim_{n \to \infty} (-1)^n = s$ is false, and thus the sequence $(-1)^n$ does not converge. 
}
\clearpage

\item (Ross 8.9)
Let $(s_n)_{n=1}^{\infty}$ be a sequence that converges.
\begin{enumerate}
\item Show that if $s_n \geq a$ for all but finitely many $n$, then $\lim s_n \geq a$.

\jg{
Since the sequence $s_n$ converges we can let $\lim s_n = s$. Therefore, for some $\epsilon > 0$, there exists $N$ such that $n > N$, that implies $|s_n - s| < \epsilon$. From the problem definition, if $s_n \geq a$ for all but finitely many $n$, this means that there exists $N_1 \in \mathbb{N}$ such that for all $n > N_1$, we have $s_n \geq a$. 

Let's suppose, for the sake of contradiction, that $s < a$. Let $\epsilon = a - s$. Since $s < a$, we have that $\epsilon > 0$. Then, we have for $\epsilon = a - s$, there exists $N_2$ such that for all $n > N_2$, $|s_n - s | < \epsilon$. Expanding the absolute values gives us $s - \epsilon < s_n < s + \epsilon$. Substituting $\epsilon = a - s$, gives $s - (a - s) < s_n < s + (a - s)$. Simplifying gives $2s - a < s_n < a$. 

From the given problem statement, we have that for all $n > N_1$, we have $s_n \geq a$. However from the second part we have for all $n > N_2$ we have $s_n < a$. If we choose $N = \max(N_1, N_2)$, then for all $n > N$ we will have $s \geq a$ and $s < a$, which is a contradiction since both cannot hold. Therefore, our assumption was false and we can conclude that if $s_n \geq a$ for all finitely many $n$, then $\lim s_n = s \geq a$. 
}

\item Show that if $s_n \leq b$ for all but finitely many $n$, then $\lim
s_n \leq b$.

\jg{
We will follow a similar structure to the above. 

Since the sequence $s_n$ converges we can let $\lim s_n = s$. Therefore, for some $\epsilon > 0$, there exists $N$ such that $n > N$, that implies $|s_n - s| < \epsilon$. From the problem definition, if $s_n \leq b$ for all but finitely many $n$, this means that there exists $N_1 \in \mathbb{N}$ such that for all $n > N_1$, we have $s_n \leq b$. 

Let's suppose, for the sake of contradiction, that $s > b$. Let $\epsilon = s - b$. Since $s > b$, we have that $\epsilon > 0$. Then, we have for $\epsilon = s - b$, there exists $N_2$ such that for all $n > N_2$, $|s_n - s | < \epsilon$. Expanding the absolute values gives us $s - \epsilon < s_n < s + \epsilon$. Substituting $\epsilon = s - b$, gives $s - (s - b) < s_n < s + (s - b)$. Simplifying gives $b < s_n < 2s - b$. 

From the given problem statement, we have that for all $n > N_1$, we have $s_n \leq b$. However from the second part we have for all $n > N_2$ we have $s_n > b$. If we choose $N = \max(N_1, N_2)$, then for all $n > N$ we will have $s \leq b$ and $s > b$, which is a contradiction since both cannot hold. Therefore, our assumption was false and we can conclude that if $s_n \leq b$ for all finitely many $n$, then $\lim s_n = s \leq b$.
}

\item Conclude if all but finitely and $s_n$ belong to $[a, b]$, then $\lim
s_n$ belongs to the interval $[a, b].$

\jg{
From parts (a) and (b), if $s_n$ belongs to $[a,b]$ for all but finitely many $n$, then we showed this implies $\lim s_n \geq a_n$ and $\lim s_n \leq b$, and thus $\lim s_n$ belongs to the interval $[a,b]$. 
}
\end{enumerate}
\end{enumerate}

\clearpage
\begin{center}
\vspace*{\fill}
{\Large End of Homework}
\vspace*{\fill}
\end{center}
\end{document}
%%%%%%%%%%%%%%%%%%%%%%%%%%%%%%%%%%%%%%%%%%%%%%%%%%%%%%%%%%%%%%%%%%%%%%
