%%%%%%%%%%%%%%%%%%%%%%%%%%%%%%%%%%%%%
% -*- LaTeX -*- %%%%%%%%%%%%%%%%%%%%%%%%%%%%%%%%%%%%%%%%%%%%%%%%%%%%%%
%
%%%%%%%%%%%%%%%%%%%%%%%%%%%%%%%%%%%%%%%%%%%%%%%%%%%%%%%%%%%%%%%%%%%%%%
%**start of header
\documentclass [10pt]{article}
\usepackage{epsfig}
\usepackage{amsmath,amsfonts,amsthm,amssymb}
\usepackage{setspace}
\usepackage{Tabbing}
\usepackage{fancyhdr}
\usepackage{lastpage}
\usepackage{extramarks}
\usepackage{chngpage}
\usepackage{graphicx,float,wrapfig}
\usepackage{amssymb}
\usepackage{mathtools}
\usepackage{esint}
\usepackage{mathrsfs}
\usepackage{cancel}
\usepackage{xcolor}  
% In case you need to adjust margins:
\topmargin=-0.45in %
\evensidemargin=0in %
\oddsidemargin=0in %
\textwidth=6.5in %
\textheight=9.0in %
\headsep=0.25in %
\newtheorem{theorem}{Theorem}[subsection]
\newtheorem{definition}[theorem]{Definition}
\newtheorem{claim}[theorem]{Claim}
\newtheorem{lemma}[theorem]{Lemma}
\newtheorem{example}[theorem]{Example}
\newtheorem{corollary}[theorem]{Corollary}
\newtheorem{proposition}[theorem]{Proposition}

\newcommand{\jg}[1]{{\color{blue} #1}}

\begin{document}
\begin{center}
{\bf Homework Problems}\\
Math 327, Winter 2025\\
Due 11:30 pm, January 15, 2025
\end{center}

\begin{center}
\jg{
    Jake Gonzales}
\end{center}

{\bf Instructions:} Please write your solution to each problem on a separate page,
and please include the
full problem statement at the top of the page. All solutions must be written in
legible handwriting
or typed (in each case, the text should be of a reasonable size).
Your solutions to all problems should be written in complete sentences, with proper
grammatical
structure.

In all cases where external resources are consulted or used, proper citation must
be given. In addition,
you must provide information on collaboration with your submission: if you worked
with others,
or consulted anyone aside from the course staff, in preparing the solutions, their
names should be
listed; if you didn't work with anyone, please indicate this.
If your solutions are not typed, you must scan your written solutions and submit
the digital copy. When submitting problems through LaTex, the LaTex source file
(.tex) must be included in the submission. \\

\jg{
\textbf{Collaborators: N/A }
}

\begin{enumerate}

\item (Ross 1.8, (a)) Prove that $n^2> n+1$ for all integers $n\geq 2$.

\jg{
We first define the extension of mathematical induction: a list $P_m, P_{m+1}, \cdots$ is true provided (i) $P_m$ is true, (ii) $P_{n+1}$ is true whenever $P_n$ is true and $n \geq m$. 

For this problem, we set $m = 2$. We define our $n$th proposition as 
\begin{align*}
    P_n : n^2 > n+1 \quad \text{for all integers} \quad  n \geq 2.  
\end{align*}
This asserts the basis of induction as $P_m = P_2 = 2^2 > 2 + 1$, which is clearly true. Suppose $P_n$ is true for any integer $n \geq 2$. We now want to verify $P_{n+1}$, 
\begin{align*}
    P_{n+1} : (n+1)^2 > (n+1) + 1.
\end{align*}
Expanding and combining terms we get 
\begin{align*}
    n^2 + 2n + 1 > n + 2. 
\end{align*}
We now substitute in our inductive hypothesis 
\begin{align*}
    (n+1)^2 &= n^2 + 2n + 1 \\
            &> (n+1) + 2n + 1 \\
            &= 3n + 2 \\
            &> n + 2 
\end{align*}
with the last inequality holding true since $n \geq 2$. Thus by mathematical induction we have proven $P_{n+1}$ holds for all integers $n \geq 2$, completing the proof. 
}
\clearpage
\item (Ross 1.8, (b)) Prove that $n! > n^2$ for $n\geq 4$.

\jg{
For this problem, we set $m=4$. We define our $n$th proposition as 
\begin{align*}
    P_n : n! > n^2 \quad \text{for all integers} \quad n \geq 4. 
\end{align*}
This asserts the basis of induction as $P_m = P_4 = 4! > 4^2 \implies 4 \cdot 3 \cdot 2 \cdot 1 > 16$, which is equivalent to $24 > 16$. Thus we can determine that our basis of induction is true. Suppose $P_n$ is true for any integer $n \geq 4$. We now want to verify $P_{n+1}$, 
\begin{align*}
    P_{n+1} : (n + 1)! > (n+1)^2.
\end{align*}
Notice $(n+1)! = (n+1) \cdot n!$. Let's now substitute in our inductive hypothesis
\begin{align*}
    (n+1)! &= (n+1) \cdot n! \\
           &> (n+1) \cdot n^2 \quad \text{by the inductive hypothesis $n! > n^2$} \\
           &= n^3 + n^2  \\
           &> n^2 + 2n + 1 \quad \text{since $n^3 > 2n + 1$ for all $n \geq 4$} \\
           &= (n+1)^2. 
\end{align*}
Thus by mathematical induction we have proven $P_{n+1}$ holds for all integers $n\geq4$, completing the proof. 

}
\clearpage
\item (Ross 2.4) Show that $(5-3^{\frac{1}{2}})^{\frac{1}{3}}$ is not a rational
number.

\jg{
Note: Below we reference and use the same structure as Example 5 from the textbook.

We can write the expression $a = \sqrt[3]{5 - \sqrt{3}}$ which means $a^3 = 5 - \sqrt{3}$ or $a^3 - 5 = \sqrt{3}$ so that $(a^3 - 5)^2 = 3$. Therefore, we have $(a^3 - 5)^2 - 3= 0$ or expanded as $a^6 - 10a^3 + 22 = 0$. So $\sqrt[3]{5 - \sqrt{3}}$ is a solution of $a^6 - 10a^3 + 22 = 0$. By Corollary 2.3 from the textbook, the only possible rational solutions are $\pm 1, \pm 2, \pm 11, \pm 22$. When $x = 1$, the left hand side equals 13, and for $x = -1$ the left hand side equals 33. For $x = 2$, the left hand side equals 6, and when $x = -2$, it equals 166. For the larger values, we can observe that a = $\sqrt[3]{5 - \sqrt{3}}$ is clearly less than 2 since 5 - $\sqrt{3} < 4$, so its cube root must be less than 2. 

We can make this clear by observing 
\begin{align*}
    11^6 - 10(11)^3 + 22 = 11(11^5 - 10 \cdot 11^2 + 2) \neq 0
\end{align*}
since the term in the parenthesis cannot be zero it is some multiple of 11 which is clearly not equal to 0. Therefore, since $\sqrt[3]{5 - \sqrt{3}}$ satisfies this equation but none of the possible rational solutions do, $\sqrt[3]{5 - \sqrt{3}}$ must be irrational.
}
\clearpage
\item (Ross 3.5)
\begin{enumerate}
\item Show that $|b| \leq a$ if and only if $-a \leq b \leq a.$

\jg{
First, we look to prove the forward direction that $|b| \leq a \implies -a \leq b \leq a$. Suppose $|b| \leq a$, then $-a \leq - |b|$. Therefore, its clear to see that $-a \leq -|b| \leq b \leq |b| \leq a$, where $-|b| \leq b \leq |b|$ holds by the definition of absolute values (definition 3.3 from the text). Thus, we have shown that $-a \leq b \leq a$. 

Now lets look at the reverse direction. Suppose $-a \leq b \leq a$. If $b \geq 0$, then $|b| = b \leq a$. If $b < 0$, then $|b| = -b \leq a$, which holds by Theorem 3.2 (i) from the textbook since we know by assumption that $-a \leq b$. Thus we have shown that $|b| \leq a$. \\
}
\item Prove that $||a|-|b|| \leq |a-b|$ for all $a, b \in \mathbb{R}$.

\jg{
From part (a), we showed that it suffices to prove $-|a-b| \leq |a| - |b| \leq |a-b|$. 

Consider $|a|$, then add and subtract $b$ to get $|a| = |a + (-b + b)| = |(a-b) + b|$. Let's look at the second inequality $|a| - |b| \leq |a-b|$. We have $|a| = |(a-b) + b|$. By the triangle inequality $|(a-b) + b| + b \leq |a-b| + |b|$. So $|a| \leq |a-b| + |b|$. Subtracting $|b|$ from both sides, we get $|a| - |b| \leq |a-b|$. 

Now consider $|b|$, then add and subtract $a$ to get $|b| = |b + (-a + a)| = |(b-a) + a|$. Let's now look at the first inequality $-|a-b| \leq |a| - |b|$. By the triangle inequality $|(b-a) + a| \leq |b-a| + |a| = |a-b| + |a|$. So $|b| \leq |a-b| + |a|$. Subtract $|b|$ from both sides: $0 \leq |a-b| + |a| - |b|$. Subtract $|a-b|$ from both sides: $-|a-b| \leq |a| - |b|$.

This completes the proof since we have shown both inequalities $-|a-b| \leq |a| - |b| \leq |a-b|$. 
}
\end{enumerate}
\clearpage
\item (Ross 3.6)
\begin{enumerate}
\item Prove that $|a+b+c| \leq |a|+|b|+|c|$ for all $a, b, c \in \mathbb{R}
$.

\jg{
We will use the hint from the textbook and apply the triangle inequality twice. 

First we group $(a+b)$. Then we apply the triangle inequality to get
\begin{align*}
    |(a+b) + c| \leq |a + b| + |c|.
\end{align*}
Then separately we apply the triangle inequality to $(a+b)$, so 
\begin{align*}
    |a+b| \leq |a| + |b|. 
\end{align*}
Putting this together we get 
\begin{align*}
    |a+b+c| = |(a+b) + c| \leq |a+b| +|c| \leq (|a| + |b|) + |c| = |a| + |b| + |c|. 
\end{align*}

Therefore, we have proven that $|a+b+c| \leq |a|+|b|+|c|$ for all $a, b, c \in \mathbb{R}$. \\
}

\item Use induction to prove
\begin{align*}
|a_1+a_2 + \cdots + a_n| \leq |a_1|+|a_2|+ \cdots + |a_n|
\end{align*}
for $n$ numbers $a_1, a_2, ..., a_n$.

\jg{
We define our nth proposition as 
\begin{align*}
    P_n : |a_1+a_2 + \cdots + a_n| \leq |a_1|+|a_2|+ \cdots + |a_n| \quad \text{for $n$ numbers $a_1, a_2, ..., a_n$}.
\end{align*}
The basis for induction $P_1 = |a_1| \leq |a_1|$ is trivially true by equality and $P_2 = |a_1 + a_2| \leq |a_1| + |a_2|$ is true by the triangle inequality. Suppose $P_n$ is true for some $n \geq 2$, we have
\begin{align*}
    |a_1 + a_2 +, \cdots, + a_n| \leq |a_1| + |a_2| + \cdots + |a_n|.
\end{align*}
Now we want to prove its true for $P_{n+1}$. If we group the first $n$ terms we have 
\begin{align*}
    |(a_1 + a_2 +, \cdots, + a_n) + a_{n+1}| \leq |a_1 + a_2 + \cdots + a_n| + |a_{n+1}|. 
\end{align*}
Using our inductive hypothesis on the first term 
\begin{align*}
    |a_1 + a_2 + \cdots + a_n| + |a_{n+1} | \leq (|a_1| + |a_2| + \cdots + |a_n|) + |a_{n+1}| = |a_1| + |a_2| + \cdots + |a_n| + |a_{n+1}|. 
\end{align*}
Therefore, by the principle of mathematical induction we have proven that $|a_1+a_2 + \cdots + a_n| \leq |a_1|+|a_2|+ \cdots + |a_n|$ for all $n \geq 1$ and for all numbers $a_1, a_2, ..., a_n$. 
}
\end{enumerate}
\clearpage
\item (Ross 3.7)
\begin{enumerate}
\item Show that $|b|< a$ if and only if $-a < b < a.$

\jg{
We will follow very closely the answer to 3.5 part (a). 

First, we look to prove the forward direction that $|b| < a \implies -a < b < a$. Suppose $|b| < a$, then $-a < -|b|$. Therefore, its clear to see that $-a < -|b| \leq b \leq |b| < a$, where $-|b| \leq b \leq |b|$ holds by the definition of absolute values (definition 3.3 from the text). Thus, we have shown that $-a < b < a$. 

Now lets look at the reverse direction. Suppose $-a < b < a$. If $b \geq 0$, then $|b| = b < a$. If $b < 0$, then $|b| = -b < a$, which holds by Theorem 3.2 (i) from the textbook since we know by assumption that $-a < b$. Thus we have shown that $|b| < a$. \\
}

\item Show that $|a-b|< c$ if and only if $b-c < a < b+c.$

\jg{
By part (a), we know that $|a-b| < c$ if and only if $-c < a-b < c$. And we know that $-c < a-b < c$ holds if and only if $b - c < a < b+c$ by $\textbf{O4}$ from the textbook. Therefore, we have proven $|a-b|< c$ if and only if $b-c < a < b+c$.  \\
}

\item Show that $|a-b|\leq c$ if and only if $b-c \leq a \leq b+c.$

\jg{
By 3.5 part (a), we know that $|a-b| \leq c$ if and only if $-c \leq a-b \leq c$. And we know that $-c \leq a-b \leq c$ holds if and only if $b-c \leq a \leq b+c$ by $\textbf{O4}$ from the textbook. Therefore, we have proven $|a-b| \leq c$ if and only if $b-c \leq a \leq b+c$. \\
}

\end{enumerate}
\clearpage
\item (Ross 3.8) Let $a, b \in \mathbb{R}.$ Show if $a \leq b_1$ for every
$b_1>b$, then $a\leq b$.

\jg{
We will prove this by contradiction. 

Suppose $a \nleq b$. By $\textbf{O1}$, if $a \nleq b$ then $b < a$. 

Now choose $b_1 = (a+b)/2$. This is strictly greater than $b$ since $b < a$, so this choice of $b_1$ satisfies $b_1 = (a+b)/2 > b$. Also, $(a+b)/2 < a$ since it is the average (or midpoint) of $a$ and $b$, and since $b < a$. Therefore, $b_1$ must be less than $a$. 

This choice of $b_1$ satisfies $b_1 > b$, but $b_1 < a$, meaning $a \nleq b_1$. This contradicts our hypothesis that $a \leq b_1$ for every $b_1 > b$. Therefore, our assumption that $a \nleq b$ must be false, so $a \leq b$. By contradiction, we have shown that if $a \leq b_1$ for every $b_1 > b$, then $a \leq b$.
}

\clearpage
\item Let $a, b>0$. Show that $(ab)^{\frac{1}{2}} \leq \frac{1}{2}(a+b).$
\end{enumerate}

\jg{
Let $a, b > 0$. We want to show that $(ab)^{\frac{1}{2}} \leq \frac{1}{2}(a+b)$. Since $a,b > 0$, both sides of the inequality are positive. Then, we can square both sides to get
\begin{align*}
    \frac{1}{4}(a+b)^2 = \frac{1}{4}(a^2 + 2ab + b^2) \geq ab.
\end{align*}
Subtracting $ab$ from each side gives 
\begin{align*}
    \frac{1}{4}(a^2 + 2ab + b^2) - ab \geq 0. 
\end{align*}
Distributing $\frac{1}{4}$ gives 
\begin{align*}
    \frac{1}{4}a^2 + \frac{1}{2}ab + \frac{1}{4}b^2 - ab \geq 0.
\end{align*}
Let's now rearrange terms as follows 
\begin{align*}
   &\frac{1}{4}a^2 - \frac{1}{2}ab + \frac{1}{4}b^2 \geq 0\\
   &\frac{1}{4}(a^2 - 2ab + b^2) \geq 0 \\
   &\frac{1}{4}(a-b)^2 \geq 0
\end{align*}
Let $c = (a-b)$. By Theorem 3.2 (iv) from the textbook, we know that $c^2 = (a-b)^2 \geq 0$ for all $c$ (or all real numbers $a$ and $b$). Therefore $\frac{1}{4}(a-b)^2$ is positive. Thus, we have proved the inequality $(ab)^{\frac{1}{2}} \leq \frac{1}{2}(a+b)$ for all $a, b > 0$. 
}

\clearpage
\begin{center}
\vspace*{\fill}
{\Large End of Homework}
\vspace*{\fill}
\end{center}

\end{document}
%%%%%%%%%%%%%%%%%%%%%%%%%%%%%%%%%%%%%%%%%%%%%%%%%%%%%%%%%%%%%%%%%%%%%%