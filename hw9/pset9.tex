%%%%%%%%%%%%%%%%%%%%%%%%%%%%%%%%%%%%%
% -*- LaTeX -*- %%%%%%%%%%%%%%%%%%%%%%%%%%%%%%%%%%%%%%%%%%%%%%%%%%%%%%
%
%%%%%%%%%%%%%%%%%%%%%%%%%%%%%%%%%%%%%%%%%%%%%%%%%%%%%%%%%%%%%%%%%%%%%%
%**start of header
\documentclass [10pt]{article}
\usepackage{epsfig}
\usepackage{amsmath,amsfonts,amsthm,amssymb}
\usepackage{setspace}
\usepackage{Tabbing}
\usepackage{fancyhdr}
\usepackage{lastpage}
\usepackage{extramarks}
\usepackage{chngpage}
\usepackage{graphicx,float,wrapfig}
\usepackage{amssymb}
\usepackage{mathtools}
\usepackage{esint}
\usepackage{mathrsfs}
\usepackage{cancel}
\usepackage{xcolor}  
\usepackage{verbatim}

\newcommand{\jg}[1]{{\color{blue} #1}}

% In case you need to adjust margins:
\topmargin=-0.45in %
\evensidemargin=0in %
\oddsidemargin=0in %
\textwidth=6.5in %
\textheight=9.0in %
\headsep=0.25in %
\newtheorem{theorem}{Theorem}[subsection]
\newtheorem{definition}[theorem]{Definition}
\newtheorem{claim}[theorem]{Claim}
\newtheorem{lemma}[theorem]{Lemma}
\newtheorem{example}[theorem]{Example}
\newtheorem{corollary}[theorem]{Corollary}
\newtheorem{proposition}[theorem]{Proposition}



\begin{document}
\begin{center}
{\bf Homework Problems}\\
Math 327, Winter 2025\\
Due 11:30 pm, March 12, 2025

\begin{center}
\jg{
    Jake Gonzales}
\end{center}


\end{center}
{\bf Instructions:} Please include the full problem statement in your submission.
All solutions must be written in legible handwriting
or typed (in each case, the text should be of a reasonable size). Your solutions to
all problems should be written in complete sentences, with proper grammatical
structure.
In all cases where external resources are consulted or used, proper citation must
be given. In addition,
you must provide information on collaboration with your submission: if you worked with others,
or consulted anyone aside from the course staff, in preparing the solutions, their
names should be
listed; if you didn't work with anyone, please indicate this.
If your solutions are not typed, you must scan your written solutions and submit
the digital copy. When submitting problems through LaTex, the LaTex source file
(.tex) must be included in the submission.


\jg{
\textbf{Collaborators: N/A }
}


\begin{enumerate}
\item Let $(a_{n, m})$ be the double sequence defined by
\begin{align*}
a_{n,m}= \begin{cases}0 & \mbox{ for }n<m, \\ -1 & \mbox{ for } n=m, \\
2^{m-n} & \mbox{ for }n>m .\end{cases}
\end{align*}
Prove that
\begin{align*}
\sum_{n=1}^{\infty} \sum_{m=1}^{\infty} a_{n, m}=-2, \quad \sum_{m=1}^{\infty} \sum_{n=1}^{\infty} a_{n, m}=0.
\end{align*}

\jg{

Let's first evaluate $\sum_{n=1}^{\infty} \sum_{m=1}^{\infty} a_{n, m}$. We can fix $n$ and sum over $m$: 
\begin{align*}
    \sum_{m=1}^{\infty} a_{n,m} = \sum_{m=1}^{n-1} a_{n,m} + a_{n,n} + \sum_{m=n+1}^\infty a_{n,m}.
\end{align*}
Using the definition of $a_{n,m}$ we have 
\begin{align*}
    \sum_{m=1}^{\infty} a_{n,m} = \sum_{m=1}^{n-1} 2^{m-n} - 1 + 0.
\end{align*}
The sum $\sum_{m=1}^{n-1} 2^{m-n}$ is a finite geometric series 
\begin{align*}
    \sum_{m=1}^{n-1} 2^{m-n} = 2^{-n} \sum_{m=1}^{n-1} 2^m =  2^{-n} (2^n - 2) = 1 - 2^{1-n}.
\end{align*}
Therefore $\sum_{m=1}^\infty a_{n,m} = (1 - 2^{1-n}) - 1 = -2^{1-n}$. Now summing over $n$ gives 
\begin{align*}
    \sum_{n=1}^{\infty} \sum_{m=1}^{\infty} a_{n, m} = \sum_{n=1}^\infty -2^{1-n} = -2 \sum_{n=1}^\infty 2^{-n}.
\end{align*}
This is another geometric series, so $\sum_{n=1}^\infty 2^{-n} = 1$. Thus, $\sum_{n=1}^{\infty} \sum_{m=1}^{\infty} a_{n, m} = -2$. 

Now, let's look at $\sum_{m=1}^{\infty} \sum_{n=1}^{\infty} a_{n, m}$. Using the definition of $a_{n,m}$, we have 
\begin{align*}
    \sum_{n=1}^{\infty} a_{n, m} = 0 - 1 + \sum_{n=m+1}^\infty 2^{m-n}. 
\end{align*}
The sum $\sum_{n=m+1}^\infty 2^{m-n}$ is an infinite geometric series, let $k=n-m$, then we have the first term as $2^{-1}$ and common ratio $r = 1/2$, so 
\begin{align*}
    \sum_{n=m+1}^\infty 2^{m-n} = \sum_{k=1}^\infty 2^{-k} = 1
\end{align*}
Therefore, $\sum_{n=1}^{\infty} a_{n, m} = -1 + 1 = 0$. Now summing over $m$: 
\begin{align*}
    \sum_{m=1}^{\infty} \sum_{n=1}^{\infty} a_{n, m} = \sum_{m=1}^\infty 0 = 0.
\end{align*}
Hence, we've shown that $\sum_{n=1}^{\infty} \sum_{m=1}^{\infty} a_{n, m}=-2$ and $\sum_{m=1}^{\infty} \sum_{n=1}^{\infty} a_{n, m}=0.$

}

\clearpage
\item
Prove that if $a_{n, m} \geq 0$ for all $n$ and $m$, then
\begin{align*}
\sum_{n=1}^{\infty} \sum_{m=1}^{\infty} a_{n, m}=\sum_{m=1}^{\infty} \sum_{n=1}^{\infty} a_{n, m}
\end{align*}
(the case $+\infty=+\infty$ may occur).


\jg{

First we will use the theorem that we defined in class (its on lecture notes 24 page 4 in the class notes): 

\textbf{Theorem.} Let $(a_{n,m})_{n,m = 1}^\infty$ be a double sequence of real numbers.
If $\sum_{m=1}^{\infty} \sum_{n=1}^{\infty} |a_{n,m}|$ converges then
$$\sum_{m=1}^{\infty} \sum_{n=1}^{\infty} a_{n,m} = \sum_{n=1}^{\infty} \sum_{m=1}^{\infty} a_{n,m}.$$

We will now consider two cases.

\textbf{Case 1:} $\sum_{n=1}^{\infty} \sum_{m=1}^{\infty} a_{n, m} < \infty$ 

Since $a_{n,m} \geq 0$ for all $n$ and $m$, we have $|a_{n,m}| = a_{n,m}$. Therefore:
\begin{align*}
\sum_{n=1}^{\infty} \sum_{m=1}^{\infty} |a_{n,m}| = \sum_{n=1}^{\infty} \sum_{m=1}^{\infty} a_{n,m} < \infty
\end{align*}

This means the double sum of absolute values converges. By the theorem above, we can conclude:
\begin{align*}
\sum_{n=1}^{\infty} \sum_{m=1}^{\infty} a_{n,m} = \sum_{m=1}^{\infty} \sum_{n=1}^{\infty} a_{n,m}
\end{align*}

\textbf{Case 2:} $\sum_{n=1}^{\infty} \sum_{m=1}^{\infty} a_{n, m} = \infty$ 

We need to show that $\sum_{m=1}^{\infty} \sum_{n=1}^{\infty} a_{n, m} = \infty$.

For any $K$ there exists $N_0$ and $M_0$ such that 
\begin{align*}
    S_{N_0, M_0} = \sum_{n=1}^{N_0} \sum_{m=1}^{M_0} a_{n,m} > K.
\end{align*}
Since $a_{n,m} \geq 0$ the partial sums $S_{N,M}$ and $T_M = \sum_{m=1}^M \sum_{n=1}^\infty a_{n,m}$ are monotonically increasing in both $N$ and $M$. For any $M \geq M_0$ we have
\begin{align*}
    T_M = \sum_{m=1}^{M} \sum_{n=1}^\infty a_{n,m} \geq T_{M_0} = \sum_{m=1}^{M_0} \sum_{n=1}^\infty a_{n,m} \geq S_{N_0, M_0} = \sum_{n=1}^{N_0} \sum_{m=1}^{M_0} a_{n,m} > K.
\end{align*}
The first inequality holds because $T_M$ includes all the terms in $T_{M_0}$ and the second holds by the non-negativity of $a_{n,m}$. Since $K$ is arbitrary, this shows that $T_M$ grows without bound as $M \to \infty$. Therefore:
\begin{align*}
    \sum_{m=1}^{\infty} \sum_{n=1}^{\infty} a_{n, m} = \lim_{M\to \infty} T_M = \infty.
\end{align*}
Hence, we have proven that if if $a_{n, m} \geq 0$ for all $n$ and $m$, then
\begin{align*}
\sum_{n=1}^{\infty} \sum_{m=1}^{\infty} a_{n, m}=\sum_{m=1}^{\infty} \sum_{n=1}^{\infty} a_{n, m}.
\end{align*}


}
\clearpage

\item (Ross 17.11) Let $f$ be a real-valued function with $\mbox{dom}(f) \subset \mathbb{R}$. Prove $f$ is continuous at $x_0$ if and only if, for every monotonic sequence $\left(x_n\right)$ in $\mbox{dom}(f)$ converging to $x_0$, we have $\lim_{n \to \infty} f(x_n)=f(x_0)$.

\jg{
We need to prove two directions. Let's start with the forward direction: if $f$ is continuous at $x_0$, then for every monotonic sequence $(x_n)$ converging to $x_0$, $\lim_{n \to \infty} f(x_n) = f(x_0)$. 

By definition 17.1 in the textbook, since $f$ is continuous at $x_0$, for every sequence $(x_n)$ in dom$(f)$ converging to $x_0$ we have $\lim_{n\to \infty} = f(x_0)$. A monotonic sequence $(x_n)$ is a special case of a sequence, therefore, if $(x_n)$ is monotonic and converges to $x_0$, the continuity of $f$ at $x_0$ implies $\lim_{n\to \infty} f(x_n) = f(x_0)$. 

Now we want to show the reverse direction: if for every monotonic sequence $(x_n)$ converging to $x_0$, $\lim_{n\to \infty} f(x_n) = f(x_0)$, then $f$ is continuous at $x_0$. 

To prove continuity at $x_0$, we have to show that for every sequence $(x_n)$ in dom$(f)$ converging to $x_0$, $\lim_{n \to \infty} f(x_n) = f(x_0)$. Suppose, for the sake of contradiction, that $f$ is not continuous at $x_0$. 

By definition 17.1, there exists a sequence $(x_n)$ in dom$(f)$ such that $\lim_{n \to \infty} x_n = x_0$, but $\lim_{n \to \infty} f(x_n) = f(x_0)$. This means there exists some $\epsilon > 0$ such that for every $N$, there is an $n > N$ with
\begin{align*}
    |f(x_n) - f(x_0) | \geq \epsilon.
\end{align*}
Using this, we can construct a subsequence $(x_{n_k})$ of $(x_n)$ such that 
\begin{align*}
    |f(x_{n_k}) - f(x_0) | \geq \epsilon \quad \text{for all} \quad k.
\end{align*}
Using the hint from the textbook, by theorem 11.4, the sequence $(x_{n_k})$ has a monotonic sequence $(x_{n_{k_i}})$. Since $(x_{n_{k_i}})$ is monotonic and converges to $x_0$ (since it is a subsequence of $(x_n)$) our initial assumption implies 
\begin{align*}
    \lim_{i\to \infty} f(x_{n_{k_i}}) = f(x_0).
\end{align*}
However, we have 
\begin{align*}
    |f(x_{n_{k_i}}) - f(x_0) | \geq \epsilon \quad \text{for all} \quad i.
\end{align*}
This contradicts that $ \lim_{i\to \infty} f(x_{n_{k_i}}) = f(x_0).$ Therefore, our assumption that $f$ is not continuous at $x_0$ must be false. Hence, $f$ is continuous at $x_0$. 


}

\clearpage
\item (Ross 17.12)
\begin{enumerate}
\item Let $f$ be a continuous real-valued function with domain $(a, b)$. Show that if $f(r)=0$ for each rational number $r$ in $(a, b)$, then $f(x)=0$ for all $x \in(a, b)$.

\jg{
We are given that $f$ is continuous on $(a,b)$ and $f(r) = 0$ for every rational number $r \in (a,b)$. We want to show that $f(x)=0$ for all $x \in (a,b)$. 

Recall, the rational numbers $\mathbb{Q}$ are dense in the real numbers $\mathbb{R}$. Therefore, for any real number $x \in (a,b)$ there exists a sequence of rational numbers $(r_n)$ in $(a,b)$ such that $r_n \to x$ as $n \to \infty$. Let $x \in (a,b)$ be an arbitrary real number. We want to show $f(x) = 0$. Since $\mathbb{Q}$ is dense in $\mathbb{R}$, there exists a sequence $(r_n)$ such that $\lim_{n\to \infty} r_n = x$. 

Since $f$ is continuous at $x$, we have $\lim_{n\to \infty} f(r_n) = f(x)$. If we know that $f(r_n) = 0$ for all $n$ (which was given), then 
\begin{align*}
    \lim_{n\to \infty} f(r_n) = \lim_{n\to \infty} = 0.
\end{align*}
Then from previous steps we have
\begin{align*}
    f(x) = \lim_{n\to \infty} f(r_n) = 0.
\end{align*}
Since $x$ was chosen arbitrarily, this holds for all $x \in (a,b)$. Thus, we've shown $f(x) = 0$ for all $x \in (a,b)$.

}

\item Let $f$ and $g$ be continuous real-valued functions on $(a, b)$ such that $f(r)=g(r)$ for each rational number $r$ in $(a, b)$. Prove $f(x)=g(x)$ for all $x \in(a, b)$.

\jg{
We are given $f$ and $g$ are continuous on $(a,b)$ and that $f(r) = g(r)$ for every rational number $r \in (a,b)$. We want to show that $f(x) = g(x)$ for all $x \in (a,b)$. 

Let $h(x) = f(x) - g(x)$. Since $f$ and $g$ are continuous, $h$ is also continuous since the difference of continuous functions are continuous. So, for every rational number $r \in (a,b)$ we have $h(r) = f(r) - g(r) = 0$ by our assumption. From part (a), since $h$ is continuous and $h(r) = 0$ for all $r \in (a,b)$, it follows that $h(x) = 0$ for all $x \in (a,b)$. 

Hence, since $h(x) = f(x) - g(x) = 0$ for all $x \in (a,b)$ we have $f(x) = g(x)$ for all $x \in (a,b)$. 
}
\end{enumerate}
\clearpage

\item Suppose $f$ is a real function defined on $\mathbb{R}$ such that
\begin{align*}
\lim _{n \to \infty}[f(x+x_n)-f(x-x_n)]=0
\end{align*}
for every $x \in \mathbb{R}$ and every sequence, $(x_n)_{n = 1}^{\infty}$ such that
$x_n \to 0$. Does this imply that $f$ is continuous?


\jg{
The condition above means that as $(x_n)$ approaches $0$, the difference $f(x+x_n)-f(x-x_n)$ approaches $0$ as $n \to \infty$. 

The question is whether this implies that $f$ is continuous. 
No. Let us show this using a counterexample. Consider the function $f$ defined as 
\begin{align*}
f(x) = \begin{cases}
1 & \text{if } x = 0, \\
0 & \text{if } x \neq 0.
\end{cases}
\end{align*}
This function is clearly discontinuous at $x = 0$. 

Let $x \in \mathbb{R}$ and $(x_n)$ be a sequence such that $x_n \to 0$. If $x \neq 0$, then for large $n$, both $x + x_n$ and $x-x_n$ are not $0$. Thus, $f(x+x_n) = f(x-x_n) = 0$, and the difference $f(x+x_n) - f(x-x_n) = 0$. 

If $x = 0$, then for all finitely many terms or non-zero $x_n$, $0+x_n = x_n \neq 0$ and $0-x_n = -x_n \neq 0$. Thus, $f(0+x_n) = f(0-x_n) = 0$, and the difference $f(0+x_n) - f(0-x_n) = 0$.

Therefore, the condition $\lim_{n \to \infty}[f(x+x_n)-f(x-x_n)]=0$ holds for every $x \in \mathbb{R}$ for every sequence $x_n \to 0$. Hence, the function $f$ satisfies the condition but it is not continuous.
}
\end{enumerate}

\clearpage
\begin{center}
\vspace*{\fill}
{\Large End of Homework}
\vspace*{\fill}
\end{center}
\end{document}
%%%%%%%%%%%%%%%%%%%%%%%%%%%%%%%%%%%%%%%%%%%%%%%%%%%%%%%%%%%%%%%%%%%%%%

