%%%%%%%%%%%%%%%%%%%%%%%%%%%%%%%%%%%%%
% -*- LaTeX -*- %%%%%%%%%%%%%%%%%%%%%%%%%%%%%%%%%%%%%%%%%%%%%%%%%%%%%%
%
%%%%%%%%%%%%%%%%%%%%%%%%%%%%%%%%%%%%%%%%%%%%%%%%%%%%%%%%%%%%%%%%%%%%%%
%**start of header
\documentclass [10pt]{article}
\usepackage{epsfig}
\usepackage{amsmath,amsfonts,amsthm,amssymb}
\usepackage{setspace}
\usepackage{Tabbing}
\usepackage{fancyhdr}
\usepackage{lastpage}
\usepackage{extramarks}
\usepackage{chngpage}
\usepackage{graphicx,float,wrapfig}
\usepackage{amssymb}
\usepackage{mathtools}
\usepackage{esint}
\usepackage{mathrsfs}
\usepackage{cancel}
\usepackage{xcolor}  
\newcommand{\hum}{\mathrm{H}}
\newcommand{\rob}{\mathrm{R}}
\newcommand{\mc}{\mathcal}
\newcommand{\mb}{\mathbb}
% In case you need to adjust margins:
\topmargin=-0.45in %
\evensidemargin=0in %
\oddsidemargin=0in %
\textwidth=6.5in %
\textheight=9.0in %
\headsep=0.25in %
\newtheorem{theorem}{Theorem}[subsection]
\newtheorem{definition}[theorem]{Definition}
\newtheorem{claim}[theorem]{Claim}
\newtheorem{lemma}[theorem]{Lemma}
\newtheorem{example}[theorem]{Example}
\newtheorem{corollary}[theorem]{Corollary}
\newtheorem{proposition}[theorem]{Proposition}

\newcommand{\jg}[1]{{\color{blue} #1}}

\begin{document}
\begin{center}
{\bf Homework Problems}\\
Math 327, Winter 2025\\
Due 11:30 pm, February 5, 2025
\end{center}

\begin{center}
\jg{
    Jake Gonzales}
\end{center}

{\bf Instructions:} Please include the full problem statement in your submission.
All solutions must be written in legible handwriting
or typed (in each case, the text should be of a reasonable size). Your solutions to
all problems should be written in complete sentences, with proper grammatical
structure.
In all cases where external resources are consulted or used, proper citation must
be given. In addition,
you must provide information on collaboration with your submission: if you worked with others,
or consulted anyone aside from the course staff, in preparing the solutions, their
names should be
listed; if you didn't work with anyone, please indicate this.
If your solutions are not typed, you must scan your written solutions and submit
the digital copy. When submitting problems through LaTex, the LaTex source file
(.tex) must be included in the submission. \\

\jg{
\textbf{Collaborators: N/A }
}

\begin{enumerate}
\item (Ross 9.3) Suppose $\lim a_n = a$, $\lim b_n = b$, and \begin{align*}
s_n = \frac{a_n^3+4a_n}{b_n^2+1}.
\end{align*}
Prove $\lim s_n = \frac{a^3+4a}{b^2+1}$ carefully, using the limit theorems.

\jg{
We are given that $\lim a_n = a$. By Theorem 9.4 from the textbook, we know that
\begin{align*}
    \lim a_n = (\lim a_n) (\lim a_n) = (a)(a) = a \cdot a = a^2.
\end{align*}
And applying Theorem 9.4 again gives
\begin{align*}
    \lim a_n^3 = (\lim a_n^2) (\lim a_n) = (a^2) (a) = a^2 \cdot a = a^3.
\end{align*}
Therefore, $\lim a_n^3 = a^3$. 

Now, by Theorem 9.2, we know that if $\lim a_n = a$, and $k = 4 \in \mathbb{R}$, $(4a_n)$ converges to $4a$. That is, $\lim (4a_n) = 4 \cdot a_n$. Furthermore, by Theorem 9.3, we know that if $(a_n^3)$ converges to $a^3$ and $(4a_n)$ converges to $4a$, then $\lim (a^3_n + 4a_n) = \lim a_n^3 + \lim 4a_n = a^3_n + 4a_n$.  

Similarly, we are given $\lim b_n = b$. Therefore, by Theorem 9.4 we know 
\begin{align*}
    \lim b_n = (\lim b_n) (\lim b_n) = (b) (b) = b \cdot b = b^2. 
\end{align*}
And clearly, $\lim (b_n^2 + 1) = b^2 + 1$. Since, $b_n^2 + 1 \neq 0$ and $b^2 + 1 \neq 0$ for all $n$, then by Theorem 9.6,
\begin{align*}
    \lim s_n = \lim \frac{a_n^3+4a_n}{b_n^2+1} = \frac{a^3+4a}{b^2+1}.
\end{align*}
Hence, we finished the proof. 
}
\clearpage
\item (Ross 9.6) Let $x_1 = 1$ and $x_{n+1} = 3x^2_n
$ for $n \geq 1$.
\begin{enumerate}
\item Show if $a = \lim x_n$, then $a = 1/3$ or $a = 0$.

\jg{
We can write $\lim x_{n+1} = \lim 3 x_n^2$.

Since $\lim x_n = a$, and by Theorem 9.2 and 9.4, we can write 
\begin{align*}
    \lim x_{n+1} = \lim 3 x_n^2 = 3 (\lim x_n)^2 = 3a^2.
\end{align*}
And since $x_{n+1}$ is just the next term in the sequence, then the limit is the same as $x_{n+1}$, so we have $\lim x_{n+1} = a$ and $\lim 3x_n^2 = 3a^2$, therefore, $3a^2 = a$. So now, solving for $a$, we can write $3a^2 - a = 0$, factoring $a(3a - 1) = 0$, then we see $a = 0$ or $3a - 1 = 0 \implies a = 1/3$.

Thus, if the limit of $x_n$ exists and is equal to $a$, then $a$ must be equal to $1/3$ or $0$. 
}
\item Does $\lim x_n$ exist? Explain.

\jg{
The sequence is defined as $x_1 = 1$ and $x_{n+1} = 3x_n^2$. We can compute a few instances of the sequence: 
\begin{align*}
    x_1 = 1, \\
    x_2 =3x_1^2 = 3(1)^2 = 3, \\
    x_3 = 3x_2^2 = 3(3)^2 = 27, \\
    x_4 = 3x_3^2 = 3(27)^2 = 2187, \\
\end{align*}
and so on. The sequence is growing to large values as $n$ grows since the sequence is defined as the square or the previous sequence value times three. For all $n \geq 1$, this will lead to increasingly large numbers. Therefore, we can conclude that the sequence is not bounded and therefore does not converge. So, $\lim x_n$ diverges to $+ \infty$.  Thus $\lim x_n$ does not exist. 
}
\item Discuss the apparent contradiction between parts (a) and (b).

\jg{
Well the apparent contradiction is that in part (a) we find a limit for the sequence $x_n$, but in part (b) we conclude that $\lim x_n$ does not exist since it diverges to $+ \infty$. This is a clear contradiction. But to be clear, the reason we arrived at this contradiction was because of the conditional statement provided in the question. We made assumption that IF $a = \lim x_n$, then $a$ must equal $1/3$ or $0$. We made no such assumption in part (b). 
}
\end{enumerate}
\clearpage

\item (Ross 9.9) Suppose there exists $N_0$ such that $s_n \leq t_n$ for all
$n>N_0$.
\begin{enumerate}
\item Prove that if $\lim s_n = +\infty$, then $\lim t_n = +\infty$.

\jg{
We are given $\lim s_n = +\infty$ meaning, by definition 9.8 from the textbook, for each $M>0$ there exists $N\geq N_0$ such that $s_n > M$ for all $n > N$. Then, clearly $t_n > M$ for all $n>N$, since $s_n \leq t_n$ for all $n$. 

Hence, this shows $\lim t_n = + \infty$.
}
\item Prove that if $\lim t_n = -\infty$, then $\lim s_n = -\infty$.

\jg{
We are given $\lim t_n = -\infty$ meaning, by definition 9.8, for each $M < 0$ there exists $N \geq N_0$ such that $t_n < M$ for all $n > N$. Then, clearly $s_n < M$ for all $n > N$, since $s_n \leq t_n$ for all $n$. 

Hence, this shows $\lim s_n = - \infty$.
}
\item Prove that if $\lim s_n$ and $\lim t_n$ exist, then $\lim s_n \leq \lim t_n$.

\jg{
First, in the previous two parts (a) and (b), we have shown the behavior of $(s_n)$ and $(t_n)$ for infinite limits. Lets then assume $(s_n)$ and $(t_n)$ converge.

Second, we showed in a previous homework (pset 3, problem 7.a or Ross 8.9.a) that assuming $(s_n)_{n=1}^{\infty}$ converges, then if $s_n \geq a$ for all but finitely many $n$, then $\lim s_n \geq a$. Using this, we are given $s_n \leq t_n$, so $(s_n - t_n) \leq 0$, therefore $\lim (s_n - t_n) \leq 0$. And, by theorems 9.2 and 9.3 from the textbook, we have $\lim s_n - \lim t_n \leq 0$. 

Hence, we've shown $\lim s_n \leq \lim t_n$. 
}
\end{enumerate}
\clearpage

\item (Ross 9.11)
\begin{enumerate}
\item Show that if $\lim s_n = +\infty$ and $\inf\{t_n : n \in \mathbb{N}\} > -\infty$, then $\lim (s_n + t_n)=+\infty$.

\jg{
We are given $\lim s_n = + \infty$, which means that for any $M_1 > 0$ there exists $N_1$ such that for all $n \geq N_1$, we have $s_n \geq M_1$. 

We want to show that $\lim (s_n + t_n) = + \infty$. This means that for any $M > 0$, there exits $N$ such that for all $n \geq N$, we have $s_n + t_n > M$. 

Let $M > 0$. Let $m = \inf \{t_n : n \in \mathbb{N} \}$, so we know that $t_n \geq m$ for all $n$. We want to show $s_n + t_n > M$, which can be rewritten as 
\begin{align*}
    s_n > M - t_n. 
\end{align*}
Since $t_n \geq m$, we have $M - t_n \geq M - m$. Thus, it suffices to show that 
\begin{align*}
    s_n > M - m.
\end{align*}
Therefore, for all $n > N_1$, we have 
\begin{align*}
    s_n + t_n > (M-m) + t_n \geq (M-m) + m = M.
\end{align*}
Let $N = N_1$, then for all $n > N$, we have $s_n + t_n > M$, and thus $\lim (s_n + t_n) = + \infty$. 
}
\item Show that if $\lim s_n = +\infty$ and $\lim t_n > -\infty$, then $\lim (s_n + t_n) =+\infty$.

\jg{
If $\lim t_n > - \infty$, then the sequence $(t_n)$ does not diverge to $- \infty$, so it must be bounded below by some finite number. Thus, if $\lim t_n > - \infty$, then $\inf\{t_n : n \in \mathbb{N}\} > -\infty$. 

Then, we can directly apply part (a). 

Hence, we've shown if $\lim s_n = +\infty$ and $\lim t_n > -\infty$, then $\lim (s_n + t_n) =+\infty$.
}
\item Show that if $\lim s_n = +\infty$ and if $(t_n)$ is a bounded sequence, then $\lim (s_n + t_n)=+\infty$.

\jg{
From the beginning of section 9 in the textbook, a sequence $(t_n)$ is a bounded sequence if the set $\{t_n : n \in \mathbb{N} \}$ is a bounded set, i.e., if there exits a constant $M$ such that $|t_n| \leq M$ for all $n$. 

This implies that $t_n \geq -M$ for all $n$, so by definition, $\inf \{t_n : n \in \mathbb{N} \} \geq -M > - \infty$. 

Then we can directly apply part (a). 

Hence, we've shown if $\lim s_n = +\infty$ and if $(t_n)$ is a bounded sequence, then $\lim (s_n + t_n)=+\infty$.
}
\end{enumerate}
\clearpage

\item (Ross 9.17)
Give a formal proof that $\lim n^2 = +\infty$ using only Definition 9.8.

\jg{
\textbf{\underline{Discussion:}}

From definition 9.8 from the textbook, for a sequence $(s_n)$, we write $\lim s_n = + \infty$ provided for each $M>0$ there is a number $N$ such that $n > N$ implies $s_n > M$. 

Therefore, to prove $\lim n^2 = + \infty$, we want to find an arbitrary $M > 0$ such that $n^2 > M$ or $n > \sqrt{M}$. So let's choose $N = \sqrt{M}$. \\

\textbf{\underline{Formal Proof:}}

Let $M > 0$ and let $N = \sqrt{M}$. Then $n > N$ implies $n > \sqrt{M}$, hence $n^2 > M$. This shows $\lim n^2 = + \infty$. 
}
\clearpage
\item (Ross 9.18)
\begin{enumerate}
\item Verify $1 + a + a^2 + \cdots + a^n = \frac{1-a^{n+1}}{1-a}$ for $a \neq 1$.

\jg{
Denote the sum above as $S_n = 1 + a + a^2 + \cdots + a^n$. We want to show $S_n = \frac{1-a^{n+1}}{1-a}$ for $a \neq 1$. 

Consider the sum below
\begin{align*}
    S_n = 1 + a + a^2 + \cdots + a^n.
\end{align*}
Now, subtract $(1-a)$ from both sides
\begin{align*}
    (1-a) S_n &= (1-a)(1 + a + a^2 + \cdots + a^n) \\
    &= (1 + a + a^2 + \cdots + a^n) - (a + a^2 + a^3 + \cdots + a^{n+1}) \\
    &= 1 + (a-a) + (a^2 - a^2) + \cdots + (a^n - a^n) - a^{n+1} \\
    &= 1 - a^{n+1}. 
\end{align*}
Therefore, by dividing $(1-a)$ from each side,
\begin{align*}
    S_n = \frac{1-a^{n+1}}{1-a}. 
\end{align*}
Hence, we've shown $1 + a + a^2 + \cdots + a^n = \frac{1-a^{n+1}}{1-a}$ for $a \neq 1$. 
}

\item Find $\lim_{n \to \infty} (1 + a + a^2 + \cdots + a^n)$ for $|a| < 1$.

\jg{
We want to find the the limit $\lim_{n \to \infty} (1 + a + a^2 + \cdots + a^n)$ for $|a| < 1$, lets write out what we proved in the previous part (a)
\begin{align*}
    1 + a + a^2 + \cdots + a^n = \frac{1-a^{n+1}}{1-a} \quad \text{for} \quad a \neq 1. 
\end{align*}
Then lets consider the limit of both sides as $n \rightarrow \infty$
\begin{align*}
    \lim (1 + a + a^2 + \cdots + a^n)= \lim \frac{1-a^{n+1}}{1-a}
\end{align*}
Since $|a| < 1$, we know that $a^{n+1} \rightarrow 0$ as $n$ grows large (this is also shown in theorem 9.7 part (b)). Therefore, the limit can be simplified to
\begin{align*}
    \lim \frac{1-a^{n+1}}{1-a} = \frac{1-0}{1-a} = \frac{1}{1-a}. 
\end{align*}
Thus, the limit is $1/(1-a)$. 
}
\item Calculate $\lim_{n\to \infty}(1 + \frac{1}{3} + \frac{1}{9} + \frac{1}{27} + \cdots + \frac{1}{3^n})$.

\jg{
Let's take two approaches to calculate this limit. First, let's write out the first few values, so we know that we start the sequence with 1, then every following element is $1/3^n$ for $n \geq 1$. Let $L$ be the limit and sum of the sequence so, 
\begin{align*}
    L = 1 + \frac{1}{3^1} + \frac{1}{3^2} + \frac{1}{3^3} + \frac{1}{3^4} = \frac{1}{3} + \frac{1}{9} + \frac{1}{27} + \frac{1}{81} = 1.4938.
\end{align*}
So we can see for larger $n$, we approach $1.5$. That would be the approximation I'd arrive at.

For a more formal approach, we can look at section 10.3 of the textbook on page 59 where they show the following fact for a geometric sequence 
\begin{align*}
    \lim a(1 + r + r^2 + \cdots + r^n) = \frac{a}{r - 1} \quad \text{for} \quad |r| < 1. 
\end{align*}
Thus, we can compute the limit by substituting $a = 1$ and $r=\frac{1}{3}$, giving $\frac{1}{1 - (1/3)}$. Hence, the limit is $3/2$ as we approximated. 
}

\item What is $\lim_{n\to \infty}(1 + a + a^2 + \cdots + a^n)$ for $a\geq 1$?

\jg{
The sequence clearly grows exponentially for $a > 1$ and linearly for $a = 1$ as $n$ grows large. For both cases, by definition 9.8 from the textbook, $\lim_{n\to \infty}(1 + a + a^2 + \cdots + a^n) = + \infty$ for $a \geq 1$. 
}
\end{enumerate}
\end{enumerate}

\clearpage
\begin{center}
\vspace*{\fill}
{\Large End of Homework}
\vspace*{\fill}
\end{center}

\end{document}
%%%%%%%%%%%%%%%%%%%%%%%%%%%%%%%%%%%%%%%%%%%%%%%%%%%%%%%%%%%%%%%%%%%%%%

