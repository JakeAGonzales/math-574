%%%%%%%%%%%%%%%%%%%%%%%%%%%%%%%%%%%%%
% -*- LaTeX -*- %%%%%%%%%%%%%%%%%%%%%%%%%%%%%%%%%%%%%%%%%%%%%%%%%%%%%%
%
%%%%%%%%%%%%%%%%%%%%%%%%%%%%%%%%%%%%%%%%%%%%%%%%%%%%%%%%%%%%%%%%%%%%%%
%**start of header
\documentclass [10pt]{article}
\usepackage{epsfig}
\usepackage{amsmath,amsfonts,amsthm,amssymb}
\usepackage{setspace}
\usepackage{Tabbing}
\usepackage{fancyhdr}
\usepackage{lastpage}
\usepackage{extramarks}
\usepackage{chngpage}
\usepackage{graphicx,float,wrapfig}
\usepackage{amssymb}
\usepackage{mathtools}
\usepackage{esint}
\usepackage{mathrsfs}
\usepackage{cancel}
\usepackage{xcolor}  

\newcommand{\jg}[1]{{\color{blue} #1}}

% In case you need to adjust margins:
\topmargin=-0.45in %
\evensidemargin=0in %
\oddsidemargin=0in %
\textwidth=6.5in %
\textheight=9.0in %
\headsep=0.25in %
\newtheorem{theorem}{Theorem}[subsection]
\newtheorem{definition}[theorem]{Definition}
\newtheorem{claim}[theorem]{Claim}
\newtheorem{lemma}[theorem]{Lemma}
\newtheorem{example}[theorem]{Example}
\newtheorem{corollary}[theorem]{Corollary}
\newtheorem{proposition}[theorem]{Proposition}



\begin{document}

\begin{center}
{\bf Homework Problems}\\
Math 327, Winter 2025\\
Due 11:30 pm, March 5, 2025
\end{center}

\begin{center}
\jg{
    Jake Gonzales}
\end{center}


{\bf Instructions:} Please include the full problem statement in your submission.
All solutions must be written in legible handwriting
or typed (in each case, the text should be of a reasonable size). Your solutions to
all problems should be written in complete sentences, with proper grammatical
structure.
In all cases where external resources are consulted or used, proper citation must
be given. In addition,
you must provide information on collaboration with your submission: if you worked with others,
or consulted anyone aside from the course staff, in preparing the solutions, their
names should be
listed; if you didn't work with anyone, please indicate this.
If your solutions are not typed, you must scan your written solutions and submit
the digital copy. When submitting problems through LaTex, the LaTex source file
(.tex) must be included in the submission. \\


\jg{
\textbf{Collaborators: N/A }
}

\begin{enumerate}

\item (Ross 14.13) We have seen that it is often a lot harder to find the value of
an infinite sum than to show it exists. Here are some sums that can be handled.

\begin{enumerate}
\item Calculate $\sum_{n=1}^{\infty}\left(\frac{2}{3}\right)^n$ and $\sum_{n=1}^{\infty}\left(-\frac{2}{3}\right)^n$.

\jg{
Observe that both series are geometric series of the form $\sum_{n=1}^\infty r^n$ where $| r | < 1$. The sum of an infinite series is given by $\frac{r}{1-r}$. We can use this to calculate the following series. First,
\begin{align*}
    \sum_{n=1}^{\infty}\left(\frac{2}{3}\right)^n = \frac{\frac{2}{3}}{1 - \frac{2}{3}} = 2. 
\end{align*}
And secondly, 
\begin{align*}
    \sum_{n=1}^{\infty}\left(-\frac{2}{3}\right)^n = \frac{-\frac{2}{3}}{1 - (-\frac{2}{3})} = -\frac{2}{5}
\end{align*}
Thus, we've shown the calculated sums. 
}


\item Prove $\sum_{n=1}^{\infty} \frac{1}{n(n+1)}=1$. Hint: Note that $\sum_{k=1}^n \frac{1}{k(k+1)}=$ $\sum_{k=1}^n\left[\frac{1}{k}-\frac{1}{k+1}\right]
$.

\jg{
We are given the hint to write $\sum_{n=1}^{\infty} \frac{1}{n(n+1)}=1$ as the telescoping series $\sum_{n=1}^{\infty} \frac{1}{n} - \frac{1}{n+1}$. 

We can expand the sum as 
\begin{align*}
    \sum_{n=1}^{N} \frac{1}{n} - \frac{1}{n+1} = \left(\frac{1}{1} - \frac{1}{2} \right) + \left(\frac{1}{2} - \frac{1}{3} \right) + \left(\frac{1}{3} - \frac{1}{4} \right) + \cdots + \left(\frac{1}{N} - \frac{1}{N+1} \right)
\end{align*}
Most of the terms cancel out, giving us
\begin{align*}
    \sum_{n=1}^{N} \frac{1}{n} - \frac{1}{n+1} = 1 - \left( \frac{1}{N+1}\right).
\end{align*}
Taking the limit as $N \rightarrow \infty$:
\begin{align*}
    \lim_{N \rightarrow \infty} \left( 1 - \frac{1}{N+1}\right) = 1- 0 = 1.
\end{align*}
Thus, we have finished the proof. 

}
\item Prove $\sum_{n=1}^{\infty} \frac{n-1}{2^{n+1}}=\frac{1}{2}$. Hint:
Note $\frac{k-1}{2^{k+1}}=\frac{k}{2^k}-\frac{k+1}{2^{k+1}}$.


\jg{
Using the hint, we can write the series as 
\begin{align*}
    \sum_{n=1}^{\infty} \frac{n-1}{2^{n+1}} = \sum_{n=1}^{\infty} \frac{n}{2^n} - \frac{n+1}{2^{n+1}}. 
\end{align*}
This is a telescoping series. Let's write out the first few terms: 
\begin{align*}
    \sum_{n=1}^{\infty} \frac{n}{2^n} - \frac{n+1}{2^{n+1}} = \left(\frac{1}{2^1} - \frac{2}{2^2} \right) + \left(\frac{2}{2^2} - \frac{3}{2^3} \right) + \left( \frac{3}{2^3} - \frac{4}{2^4}\right) + \cdots \left( \frac{1}{2} - \frac{N+1}{2^{N+1}} \right)
\end{align*}
Taking the limit as $N \rightarrow \infty$: 
\begin{align*}
    \lim_{N\rightarrow \infty} \left( \frac{1}{2} - \frac{N+1}{2^{N+1}} \right) = \frac{1}{2} - 0 = \frac{1}{2}
\end{align*}

}
\item Use (c) to calculate $\sum_{n=1}^{\infty} \frac{n}{2^n}$.

\jg{

We can write 
\begin{align*}
   \sum_{n=1}^{\infty}  \frac{n-1}{2^{n+1}} = \sum_{n=1}^{\infty} \left( \frac{n}{2^{n+1}} - \frac{1}{2^{n+1}}\right). 
\end{align*}
Splitting the series into two parts using part(c), 
\begin{align*}
    \sum_{n=1}^{\infty} \frac{n}{2^{n+1}} - \sum_{n=1}^{\infty}  \frac{1}{2^{n+1}} = \frac{1}{2}.
\end{align*}
For the second sum $\sum_{n=1}^{\infty}  \frac{1}{2^{n+1}}$, we can factor out $\frac{1}{2}$ to get $\frac{1}{2} \sum_{n=1}^{\infty} \frac{1}{2^n}$, and this is a geometric series with first term $\frac{1}{2}$ and common ratio $r=\frac{1}{2}$, then
\begin{align*}
    \sum_{n=1}^{\infty} \frac{1}{2^n} = \frac{\frac{1}{2}}{1 - \frac{1}{2}} = 1. 
\end{align*}
Thus, 
\begin{align*}
    \sum_{n=1}^{\infty}  \frac{1}{2^{n+1}} = \frac{1}{2} = \frac{1}{2} \cdot 1 = \frac{1}{2}. 
\end{align*}
From the second step we now have: 
\begin{align*}
    \sum_{n=1}^{\infty} \frac{n}{2^{n+1}} - \frac{1}{2} = \frac{1}{2}.
\end{align*}
Adding 
\begin{align*}
    \sum_{n=1}^{\infty} \frac{n}{2^{n+1}}  = 1.
\end{align*}
We know that $\sum_{n=1}^{\infty} \frac{n}{2^{n+1}} = \frac{1}{2} \sum_{n=1}^{\infty} \frac{n}{2^n}$. From previous step we know $\frac{1}{2} \sum_{n=1}^{\infty} \frac{n}{2^n} = 1$. Thus, 
\begin{align*}
    \sum_{n=1}^{\infty} \frac{n}{2^n} = 2. 
\end{align*}

}

\end{enumerate}
\clearpage

\item (Ross 15.7)
\begin{enumerate}
\item Prove if $(a_n)$ is a decreasing sequence of real numbers and if $\sum
a_n$ converges, then $\lim_{n \to \infty} n a_n=0.$ Hint: Consider $|a_{N+1}+a_{N+2}+ \cdots + a_n|$ for suitable $N$.

\jg{
Since $\sum a_n$ converges, by the Cauchy criterion, for any $\epsilon > 0$, there exists an integer $N$ such that for all $n \geq m \geq N$, it satisfies 
\begin{align*}
    \left| \sum_{k=m}^n a_k  \right| < \frac{\epsilon}{2}. 
\end{align*}
Here, since $(a_n)$ is given as decreasing non-negative, we can drop the absolute value and write this as 
\begin{align*}
     \sum_{k=m}^n a_k  < \frac{\epsilon}{2}. 
\end{align*}
In particular, we have that $n > N$ implies 
\begin{align*}
    a_{N+1} + \cdots + a_n < \frac{\epsilon}{2}.
\end{align*}
We fix $n>N$. Since $(a_n)$ is decreasing, the smallest term in the sum $a_{N+1}+a_{N+2}+ \cdots + a_n$ is $a_n$. Therefore, we can bound the sum from below: 
\begin{align*}
    \sum_{k=N+1}^n a_k \geq (n-N)a_n
\end{align*}
Combining the previous inequalities we get
\begin{align*}
    (n-N)a_n < \frac{\epsilon}{2}. 
\end{align*}
Then we can write 
\begin{align*}
    a_n < \frac{\epsilon}{2(n-N)}.
\end{align*}
Multiplying both sides by $n$ and simplifying: 
\begin{align*}
    n a_n < \frac{\epsilon}{2} \frac{n}{n-N}. 
\end{align*}
For any $n > 2N$, we have $n - N > \frac{n}{2}$. This gives  
\begin{align*}
    \frac{n}{n-N} < 2.
\end{align*}
Substituting this in gives
\begin{align*}
    n a_n < \frac{\epsilon}{2} \cdot 2 = \epsilon.
\end{align*}
Therefore, we have shown that for any $\epsilon > 0$, there exits $N$ such that for all $n > 2N$: 
\begin{align*}
    n a_n < \epsilon.
\end{align*}
Thus, this proves that $\lim_{n \rightarrow \infty} n a_n = 0$. 
}

\item Use (a) to give another proof that $\sum \frac{1}{n}$ diverges.

\jg{
From part (a), we had that $(a_n)$ is a decreasing sequence and we can observe that $\sum \frac{1}{n}$ is also decreasing since $\frac{1}{n+1} < \frac{1}{n}$ for all $n \geq 1$. We proved that if a sequence is decreasing and converges, then $\lim_{n\rightarrow \infty} n a_n = 0$. We want to use this to show that $\sum \frac{1}{n}$ diverges. Notice, if $\sum \frac{1}{n}$ converges, by part (a), we must have: 
\begin{align*}
    \lim_{n\rightarrow \infty} n a_n = 0. 
\end{align*}
Substituting $a_n = \frac{1}{n}$, 
\begin{align*}
    \lim_{n\rightarrow \infty} n \cdot \frac{1}{n} = \lim_{n\rightarrow \infty} 1 = 1. 
\end{align*}
Thus, the limit of $\sum \frac{1}{n}$ does not equal 0, which contradicts the proof in part (a). Hence, we've shown $\sum \frac{1}{n}$ diverges. 
}


\end{enumerate}
\clearpage



\item Let $(a_n)_{n=1}^{\infty}$ and $(b_n)_{n=1}^{\infty}$ be sequences of real
numbers. Define the sequence
\begin{align*}
c_n := \sum_{k=1}^n a_kb_{n-k}.
\end{align*}
Prove that if $\sum_n a_n$ and $\sum_n b_n$ converge absolutely, then $\sum_n c_n$
converges absolutely.


\jg{
We are given that $\sum a_n$ and $\sum b_n$ converge absolutely, which means: 
\begin{align*}
    \sum_{n=1}^\infty |a_n| < \infty \quad \text{and} \quad \sum_{n=1}^\infty |b_n| < \infty. 
\end{align*}
Our goal is to show that $\sum_{n=1}^\infty |c_n| < \infty$, \textit{i.e.,} that $\sum c_n$ converges absolutely. 

For each $n$, we have 
\begin{align*}
    |c_n| = \left|\sum_{k=1}^n a_kb_{n-k}\right|.
\end{align*}
By the triangle inequality, we get
\begin{align*}
    |c_n| \leq \sum_{k=1}^n |a_k| \cdot |b_{n-k}|. 
\end{align*}
To bound $\sum_{n=1}^\infty |c_n|$ it suffices to bound $\sum_{n=1}^\infty \sum_{k=1}^n |a_k| \cdot |b_{n-k}|$. 

Setting $m = n-k$, we can rewrite the sum to 
\begin{align*}
    \sum_{k=1}^\infty \sum_{m=1}^\infty |a_k| \cdot |b_{m}|.
\end{align*}
For each fixed $k$ above, $m=n-k$ ranges over all non-negative integers. We can separate this into the product of two independent sums: 
\begin{align*}
    \sum_{k=1}^\infty \sum_{m=1}^\infty |a_k| \cdot |b_{m}| = \left( \sum_{k=1}^\infty |a_k| \right) \left( \sum_{m=1}^\infty |b_m| \right)
\end{align*}
This is valid since both sums are finite by the assumption of absolute convergence. Now, since $\sum_{n=1}^\infty |a_n| < \infty$ and $\sum_{n=1}^\infty |b_n| < \infty$ their product is also finite: 
\begin{align*}
    \left( \sum_{k=1}^\infty |a_k| \right) \left( \sum_{m=1}^\infty |b_m| \right) < \infty.
\end{align*}
Therefore, it follows that:
\begin{align*}
    \sum_{n=1}^\infty |c_n| \leq \left( \sum_{k=1}^\infty |a_k| \right) \left( \sum_{m=1}^\infty |b_m| \right) < \infty.
\end{align*}
Hence, $\sum_{n=1}^\infty |c_n| < \infty$, which proves $\sum_{n=1}^\infty |c_n|$ converges absolutely. 
}



\clearpage
\item (Ross 16.3) Suppose $\sum a_n$ and $\sum b_n$ are convergent series of nonnegative numbers. Show that if $a_n \leq b_n$ for all $n$ and if $a_n<b_n$ for at least one $n$, then $\sum a_n<\sum b_n$.

\jg{
Let $A = \sum_{n=1}^\infty$ and $B = \sum_{n=1}^\infty$. Since $\sum a_n$ and $\sum b_n$ are given as convergent series of nonnegative numbers, $A$ and $B$ are finite and nonnegative. 

Given that we are seeking to show that $\sum a_n<\sum b_n$, we want to look at this new series $\sum_{n=1}^\infty (b_n - a_n)$. Since $a_n \leq b_n$ for all $n$, each term $b_n - a_n$ is nonnegative, so $b_n - a_n \geq 0$ for all $n$. We know that since $\sum a_n$ and $\sum b_n$ converge, then $\sum_{n=1}^\infty (b_n - a_n)$ converges. Also, by linearity of convergent series, we can write
\begin{align*}
    \sum_{n=1}^\infty (b_n - a_n) = \sum_{n=1}^\infty b_n - \sum_{n=1}^\infty  a_n = B - A. 
\end{align*}
Its given that there exists at least one $n$ such that $a_n < b_n$, then $b_n - a_n > 0$. This implies the sum $\sum_{n=1}^\infty (b_n - a_n)$ is strictly positive. That is, $\sum_{n=1}^\infty (b_n - a_n) > 0$. Then clearly we can write: 
\begin{align*}
    B - A = \sum_{n=1}^\infty (b_n - a_n) > 0. 
\end{align*}
Therefore, $B - A > 0$. Hence, $\sum a_n < \sum b_n$. 
}



\clearpage
\item Let $\left(a_n\right)$ be a sequence of real numbers, and define $s_m:= \sum_{n=1}^m a_n$. For each $M$ define $\sigma_M=\frac{1}{M} \left(s_1+s_2+\cdots+s_M\right)$. Prove that if $\lim_{n \to \infty} n a_n =0$ and $(\sigma_M)$ converges, then $\sum_n a_n$ converges.

\jg{
We assume (from problem description) that $(\sigma_m)$ converges to some limit $L$ i.e., $\lim_{M\rightarrow \infty} \sigma_M = L$. 

We want to show that $\sum_n a_n$ converges, i.e., $\lim_{m\rightarrow \infty} s_m$ exists. 

We can express $s_M$ in terms of $\sigma_M$ as follows
\begin{align*}
    \sigma_M = \frac{1}{M} (s_1 + \cdots + s_M) 
\end{align*}
and $\sigma_{M-1}$ as 
\begin{align*}
    \sigma_{M-1} \frac{1}{M-1} (s_1 + \cdots + s_{M-1})
\end{align*}
Multiplying by $M$ and $M-1$, respectively, we get
\begin{align*}
    M \sigma_M = (s_1 + \cdots + s_M) \\
    (M-1) \sigma_{M-1} = (s_1 + \cdots + s_{M-1})
\end{align*}
We can subtract the two equations to get
\begin{align*}
    M \sigma_M - (M-1)\sigma_{M-1} = s_M.
\end{align*}
Or rewritten,
\begin{align*}
    s_M = M(\sigma_M - \sigma_{M-1}) + \sigma_{M-1}. 
\end{align*}
Now let's look at the definition of $\sigma_M$, we have 
\begin{align*}
    \sigma_M - \sigma{M-1} = \frac{1}{M}(s_1 + \cdots + s_M ) - \frac{1}{M-1} (s_1 + \cdots + s_{M-1}). 
\end{align*}
Simplifying, 
\begin{align*}
    \sigma_M - \sigma_{M-1} = \frac{1}{M} s_M + \left( \frac{1}{M} - \frac{1}{M-1}\right) (s_1 + \cdots + s_{M-1}). 
\end{align*}
Substituting in $s_M = M(\sigma_M - \sigma_{M-1}) + \sigma_{M-1}$, 
\begin{align*}
    \sigma_M - \sigma_{M-1} = \frac{1}{M} (M \sigma_M - (M-1) \sigma_{M-1} ) + \left( \frac{1}{M} - \frac{1}{M-1} \right) (s_1 + \cdots + s_{M-1})
\end{align*}
Distributing $\frac{1}{M}$ and simplifying,
\begin{align*}
    \sigma_M - \sigma_{M-1} = \sigma_M - \frac{M-1}{M} \sigma_{M-1} + \left( \frac{1}{M} - \frac{1}{M-1}  \right) (s_1 + \cdots + s_{M-1}) 
\end{align*}
And finally this simplifies to 
\begin{align*}
    \sigma_M - \sigma_{M-1} = \frac{\sigma_M}{M} + \left( \frac{1}{M} - \frac{1}{M-1}  \right) (s_1 + \cdots + s_{M-1}) .
\end{align*}

Note: I tried to establish the relationship between terms in the sequence $s_M$ and the sequence $(\sigma_M)$. Started with their definitions, derived some relationship, and manipulated with the goal of finding a way to show that if $(\sigma_M)$ and $\lim_{n\rightarrow \infty} n a_n = 0$ then $s_M$ converges. I figured you could do this by showing that the difference in $s_m$ and $s_{m-1}$ then we could prove convergence. But I unfortunately started this homework late, and I do not see a clear way forward. 

}

\end{enumerate}
\end{document}
%%%%%%%%%%%%%%%%%%%%%%%%%%%%%%%%%%%%%%%%%%%%%%%%%%%%%%%%%%%%%%%%%%%%%%

