%%%%%%%%%%%%%%%%%%%%%%%%%%%%%%%%%%%%%
% -*- LaTeX -*- %%%%%%%%%%%%%%%%%%%%%%%%%%%%%%%%%%%%%%%%%%%%%%%%%%%%%%
%
%%%%%%%%%%%%%%%%%%%%%%%%%%%%%%%%%%%%%%%%%%%%%%%%%%%%%%%%%%%%%%%%%%%%%%
%**start of header
\documentclass [10pt]{article}
\usepackage{epsfig}
\usepackage{amsmath,amsfonts,amsthm,amssymb}
\usepackage{setspace}
\usepackage{Tabbing}
\usepackage{fancyhdr}
\usepackage{lastpage}
\usepackage{extramarks}
\usepackage{chngpage}
\usepackage{graphicx,float,wrapfig}
\usepackage{amssymb}
\usepackage{mathtools}
\usepackage{xcolor}  
\usepackage{esint}
\usepackage{mathrsfs}
\usepackage{cancel}
% In case you need to adjust margins:
\topmargin=-0.45in %
\evensidemargin=0in %
\oddsidemargin=0in %
\textwidth=6.5in %
\textheight=9.0in %
\headsep=0.25in %
\newtheorem{theorem}{Theorem}[subsection]
\newtheorem{definition}[theorem]{Definition}
\newtheorem{claim}[theorem]{Claim}
\newtheorem{lemma}[theorem]{Lemma}
\newtheorem{example}[theorem]{Example}
\newtheorem{corollary}[theorem]{Corollary}
\newtheorem{proposition}[theorem]{Proposition}

\newcommand{\jg}[1]{{\color{blue} #1}}

\begin{document}
\begin{center}
{\bf Homework Problems}\\
Math 327, Winter 2025\\
Due 11:30 pm, February 20, 2025
\end{center}

\begin{center}
\jg{
    Jake Gonzales}
\end{center}

{\bf Instructions:} Please include the full problem statement in your submission.
All solutions must be written in legible handwriting
or typed (in each case, the text should be of a reasonable size). Your solutions to
all problems should be written in complete sentences, with proper grammatical
structure.
In all cases where external resources are consulted or used, proper citation must
be given. In addition,
you must provide information on collaboration with your submission: if you worked with others,
or consulted anyone aside from the course staff, in preparing the solutions, their
names should be
listed; if you didn't work with anyone, please indicate this.
If your solutions are not typed, you must scan your written solutions and submit
the digital copy. When submitting problems through LaTex, the LaTex source file
(.tex) must be included in the submission.

\jg{
\textbf{Collaborators: N/A }
}

\begin{enumerate}
\item (Ross 11.2)
Consider the sequences defined as follows:
$$
a_n=(-1)^n, \quad b_n=\frac{1}{n}, \quad c_n=n^2, \quad d_n=\frac{6 n+4}{7 n-3} .
$$
\begin{enumerate}
\item For each sequence, give an example of a monotone subsequence.

\jg{
\begin{itemize}
    \item $a_n = (-1)^n$ is a sequence that alternates between $-1$ and $1$. An example of a monotone subsequence is either the constant sequence $1, 1, \cdots$ for all even indexed terms, i.e. $a_{2k}$ or $-1, -1, \cdots$ for all odd indexed terms, i.e. $a_{2k-1}$ of the sequence $a_n$. 

    \item $b_n = \frac{1}{n}$ is a strictly decreasing sequence. Therefore, any subsequence of $b_n$ is monotone and the sequence itself is monotone.

    \item $c_n = n^2$ is strictly increasing. Therefore, any subsequence of $c_n$ is monotone and the sequence itself is monotone. 

    \item $d_n=\frac{6 n+4}{7 n-3}$ clearly converges to $\frac{6}{7}$ and is monotone decreasing for $n \geq 1$. Therefore, the sequence itself is monotone. 
\end{itemize}
}
\item For each sequence, give its set of subsequential limits.

\jg{
\begin{itemize}
    \item $a_n = (-1)^n$: The subsequential limits are $-1$ and $1$, thus the set if $\{ -1, 1\}$. 
    \item $b_n = \frac{1}{n}$: The sequence converges to $0$, so the only subsequential limit is 0, thus, the set is $\{ 0 \}$. 
    \item $c_n = n^2$: The sequence diverges to $+ \infty$, so the only subsequential limit is $+ \infty$, thus, the set is $\{ \infty \}$. 
    \item  $d_n=\frac{6 n+4}{7 n-3}$: The sequence converges to $\frac{6}{7}$, so the only subsequential limit is $\frac{6}{7}$, thus the set is $\{ \frac{6}{7} \}$. 
\end{itemize}
}

\item For each sequence, give its $\limsup$ and $\liminf$.

\jg{
\begin{itemize}
    \item $a_n = (-1)^n$: The largest subsequential limit is 1 and smallest is -1. Thus $\limsup a_n = 1$ and $\liminf a_n = -1$. 
    \item $b_n = \frac{1}{n}$: The sequence converges to 0, so $\limsup b_n = \liminf b_n = 0$. 
    \item $c_n = n^2$: The sequence diverges to $+ \infty$ so $\limsup c_n = \liminf c_n = + \infty$.
    \item $d_n=\frac{6 n+4}{7 n-3}$: The sequence converges to $\frac{6}{7}$, so $\limsup d_n = \liminf d_n  = \frac{6}{7}$. 
\end{itemize}
}

\item Which of the sequences converges? diverges to $+\infty$ ? diverges to $-\infty$ ?

\jg{
\begin{itemize}
\item $a_n = (-1)^n$: The sequence does not converge, or diverge to $+ \infty$ or $- \infty$. It oscillates between $-1$ and $1$. 

\item $b_n = \frac{1}{n}$: The sequence converges to 0. 

\item $c_n = n^2$: The sequence diverges to $+ \infty$. 

\item $d_n=\frac{6 n+4}{7 n-3}$: The sequence converges to $\frac{6}{7}$.
\end{itemize}

}
\item Which of the sequences is bounded?
\jg{
\begin{itemize}
\item $a_n = (-1)^n$: The sequence is bounded, as $|a_n| \leq 1$ for all $n$. 

\item $b_n = \frac{1}{n}$: The sequence is bounded, as $|b_n| \leq 1$ for all $n$. 

\item $c_n = n^2$: The sequence is unbounded, as $c_n \rightarrow + \infty$.

\item $d_n=\frac{6 n+4}{7 n-3}$: The sequence is bounded, between the limit $\frac{6}{7}$ for large $n$ and $\frac{5}{2}$ for $n=1$. 
\end{itemize}
}

\clearpage
\end{enumerate}
\item (Ross 11.7) Let $(r_n)_{n=1}^{\infty}$ be an enumeration of the set, $\mathbb{Q}$, of all rational numbers. Show there exists a subsequence, $(r_{n_k})$
such that $\lim_{k \to \infty} r_{n_k} = +\infty$.

\jg{
We are given that $(r_n)$ is an enumeration of rational numbers $\mathbb{Q}$, meaning that every rational number appears at least once in the sequence. By the density of $\mathbb{Q}$, for any real number $M$, there is always a larger rational number (or infinitely many) greater than $M$. 

Now, we want to show that there exists a subsequence $(r_{n_k})$ such that $\lim_{k \to \infty} +\infty$. To do this, we need to show that the sequence $(r_n)$ is unbounded above. Meaning by definition for any real number $M$, there exists a term in the sequence $r_n$ such that $r_n > M$. Since $(r_n)$ is an enumeration of $\mathbb{Q}$, there are rational numbers greater than $M$ that must be in the sequence $(r_n)$. Therefore, $(r_n)$ is unbounded above. 

We now apply \textbf{Theorem 11.2 (ii)} which states that if a sequence is unbounded above, it has a subsequence with limit $+ \infty$. We directly apply this, so since $(r_n)$ is unbounded above, there exists a subsequence $(r_{n_k})$ such that 
\begin{align*}
    \lim_{k \to \infty} r_{n_k} =  +\infty. 
\end{align*}
Note: the problem is probably done here since we have shown there exists a subsequence $(r_{n_k})$
such that $\lim_{k \to \infty} r_{n_k} = +\infty$, but I will continue with the proof and construction just to be explicit. 

Following the proof structure for Theorem 11.2 from the book, first, choose $n_1$ to be an index such that $r_{n_1} > 1$. This is possible since $\mathbb{Q}$ contains rationals larger than 1, and $(r_n)$ enumerates all of $\mathbb{Q}$.

For $k \geq 2$, choose $n_k > n_{k-1}$ such that $r_{n_k} > \max\{r_{n_{k-1}}, k\}$. This is possible since $(r_n)$ is unbounded above.

The subsequence $(r_{n_k})$ constructed this way is strictly increasing and satisfies $r_{n_k} > k$ for all $k$. So for this subsequence, as $k \rightarrow \infty$, we have $r_{n_k} \rightarrow \infty$. Therefore, 
\begin{align*}
   \lim_{k \to \infty} r_{n_k} = +\infty. 
\end{align*}
}

\clearpage
\item (Ross 12.4) Show $\lim \sup \left(s_n+t_n\right) \leq \limsup s_n+\limsup
t_n$ for bounded sequences $\left(s_n\right)$ and $\left(t_n\right)$. 

Hint: First show
\begin{align*}
\sup \left\{s_n+t_n: n>N \right \} \leq \sup \left\{s_n: n>N\right\}+\sup \left\{t_n: n>N\right\}. 
\end{align*}
Then apply Exercise 9.9 (c). 

\jg{
We will follow the hint and first show that 
\begin{align*}
\sup \left\{s_n+t_n: n>N \right \} \leq \sup \left\{s_n: n>N\right\}+\sup \left\{t_n: n>N\right\}. 
\end{align*}
For any $n > N$ we have by definition
\begin{align*}
    s_n \leq \sup \{ s_k : k > N \} \quad \text{and} \quad t_n \leq \sup \{ t_k : k > N\}. 
\end{align*}
Adding the inequalities we get 
\begin{align*}
    s_n + t_n \leq \sup \{ s_k : k > N \} + \leq \sup \{ t_k : k > N\}.
\end{align*}
The right hand side of the above inequality is an upper bound for $s_n + t_n$ for all $n > N$. Therefore, 
\begin{align*}
\sup \left\{s_n+t_n: n>N \right \} \leq \sup \left\{s_n: n>N\right\}+\sup \left\{t_n: n>N\right\}. 
\end{align*}
Recall, Exercise Ross 9.9 (c) states that if $s_n \leq t_n$ for all $n > N_0$ and $\lim s_n$ and $\lim t_n$ exist, then $\lim s_n \leq \lim t_n$. Using this logic, we will take the limit of the above inequality. By \textbf{Definition 10.6} from the textbook we know that 
\begin{align*}
    \lim \sup s_n = \lim_{N \rightarrow \infty} \sup \{ s_n : n > N \}.
\end{align*}
So then, taking the limit as $n \rightarrow \infty$ on both sides gives us 
\begin{align*}
\lim_{N \rightarrow \infty} \sup \left\{s_n+t_n: n>N \right \} \leq \lim_{N \rightarrow \infty} \sup \left\{s_n: n>N\right\} + \lim_{N \rightarrow \infty} \sup \left\{t_n: n>N\right\}. 
\end{align*}
Hence, for bounded sequences $(s_n)$ and $(t_n)$ we shown that
\begin{align*}
    \lim \sup \left(s_n+t_n\right) \leq \limsup s_n+\limsup
t_n.
\end{align*}
}
\clearpage
\item (Ross 12.6) Let $\left(s_n\right)$ be a bounded sequence, and let $k$ be a nonnegative real number.
\begin{enumerate}
\item Prove $\lim \sup \left(k s_n\right)=k \cdot \lim \sup s_n$.

\jg{
First let's define the limit superior as 
\begin{align*}
    \lim \sup s_n = \lim_{N \rightarrow \infty} \sup \{ s_n : n > N \} = \sup S
\end{align*}
where $S$ is the set of all subsequential limits of $(s_n)$. Let $L = \lim \sup s_n$, then $L = \lim \sup \{ s_n : n > N\}$. Since $k$ is a nonnegative real number then multiplying $k$ preservers the order of the sequence. Therefore 
\begin{align*}
    \sup \{ k s_n : n > N\} = k \cdot \sup \{ s_n : n > N \}. 
\end{align*}
If we take the limit as $N \rightarrow \infty$ then 
\begin{align*}
    \lim \sup (k s_n) = \lim_{N \rightarrow \infty} k \cdot \sup \{ s_n : n > N \} = k \cdot \lim_{N \rightarrow \infty} \sup \{ s_n : n > N \}. 
\end{align*}
By definition, $\lim_{N \rightarrow \infty} \sup \{s_n : n > N \} = L$ so, 
\begin{align*}
    \lim \sup (k s_n) = k \cdot L. 
\end{align*}
Hence, $\lim \sup (k s_n) = k \cdot \lim \sup s_n$ for $k \geq 0$. 
}
\item Do the same for $\liminf$. 

\jg{
First let's define the limit infimum as 
\begin{align*}
    \lim \inf s_n = \lim_{N \rightarrow \infty} \inf \{ s_n : n > N \} = \inf S
\end{align*}
where $S$ is the set of all subsequential limits of $(s_n)$. 

Let $l = \lim \inf s_n$, then $l = \lim_{N \rightarrow \infty} \inf \{s_n : n > N \}$. Since $k$ is a nonnegative real number then multiplying $k$ preserves teh order of the sequence. Therefore, 
\begin{align*}
    \inf \{ k s_n : n > N \} = k \cdot \inf \{ s_n : n > N \}. 
\end{align*}
If we take the limit as $N \rightarrow \infty$ then 
\begin{align*}
    \lim \in (k s_n) = \lim_{N \rightarrow \infty} k \cdot \inf \{ s_n : n > N \} = k \cdot \lim_{N \rightarrow \infty} \inf \{ s_n : n > N \}. 
\end{align*}
By definition, $\lim_{N \rightarrow \infty} \inf \{s_n : n > N \} = l$, so 
\begin{align*}
    \lim \inf (k s_n) = k \cdot l. 
\end{align*}
Hence, $\lim \inf k s_n = k \cdot \lim \inf s_n$ for $k \geq 0$. 
}

\item What happens in (a) and (b) if $k<0$ ?

\jg{
When $k<0$, multiplying by $k$ would reverse the order of the sequence. So we get 
\begin{align*}
    \sup \{ k s_n : n > N \} = k \cdot \inf \{ s_n : n > N \}
\end{align*}
Taking the limit 
\begin{align*}
    \lim \sup k s_n = \lim_{N \rightarrow \infty} k \cdot \inf \{ s_n : n > N \} = k \cdot \lim_{N \rightarrow \infty} \inf \{ s_n : n > N \} 
\end{align*}
By definition, $\lim_{N \rightarrow \infty} \inf \{ s_n : n > N \} = l$ so
\begin{align*}
    \lim \sup (k s_n) = k \cdot l
\end{align*}
Similarly, for (b) we can arrive at the fact that $\lim \inf k s_n = k \cdot L$. 

So when $k < 0$, they are swapped since $k$ reverses the order of the sequence. 
}
\end{enumerate}
\clearpage

\item (Ross 12.10) Prove $\left(s_n\right)$ is bounded if and only if $\limsup|
s_n|<+\infty$.

\jg{
Let's do the forward direction: we want to prove if $(s_n)$ bounded, then $\limsup |s_n| < + \infty$. 

Since $(s_n)$ is bounded, there exists $M > 0$ such that $|s_n| \leq M$ for all $n \in \mathbb{N}$. The $\limsup$ of a sequence $(|s_n|)$ is 
\begin{align*}
    \limsup |s_n| = \lim_{N \rightarrow \infty} \sup \{ |s_n| : n > N \}.
\end{align*}
Since $(s_n)$ is bounded, $|s_n| \leq M$ for all $n$. This means that for any $N$, the set $\{|s_n| : n > N \}$ is also bounded above by $M$. Therefore,
\begin{align*}
    \sup \{ |s_n| : n > N \} \leq M \quad \text{for all} \quad N.
\end{align*}
If we take the limit we get
\begin{align*}
    \limsup |s_n| = \lim_{N \rightarrow \infty} \sup \{ |s_n| : n > N \} \leq M. 
\end{align*}
Since $M$ is a finite real number, $\limsup |s_n|$ is also finite. Hence, $\limsup |s_n| < + \infty$. 

Now let's prove the reverse direction: If $\limsup |s_n| < + \infty$, then $(s_n)$ is bounded. 

Let $L = \limsup |s_n|$. So by our assumption we have $L < + \infty$. 

By the definition we know that
\begin{align*}
L = \lim_{N \rightarrow \infty} \sup \{ |s_n| : n > N \}
\end{align*}

Since this limit exists and equals $L$, for any $\epsilon > 0$, there exists some $N_0$ such that for all $N \geq N_0$: 
\begin{align*}
    \sup \{|s_n| : n > N \} < L + \epsilon
\end{align*}
Let $\epsilon = 1$. This means that for all $n > N_0$ we have $|s_n| < L+1$. Then if we consider the finite set of terms $\{ |s_1|, \cdots, |s_{N_0}|\}$ and let $M_0 = \text{max}\{ |s_1|, \cdots, |s_{N_0}|\}$. Then if we set $M = \text{max} \{M_0, L+1\}$, then $|s_n| \leq M$ for all $n$. Therefore, $(s_n)$ is bounded. 

}

\clearpage
\item (Ross 12.12) Let $\left(s_n\right)$ be a sequence of nonnegative numbers,
and for each $n$ define $\sigma_n=\frac{1}{n}\left(s_1+s_2+\cdots+s_n\right)$.
\begin{enumerate}
\item Show
\begin{align*}
\liminf s_n \leq \liminf \sigma_n \leq \limsup \sigma_n \leq \limsup s_n .
\end{align*}
Hint: For the last inequality, show first that $M>N$ implies
$$
\sup \left\{\sigma_n: n>M\right\} \leq \frac{1}{M}\left(s_1+s_2+\cdots+s_N\right)+\sup \left\{s_n: n>N\right\} \text {. }
$$

\jg{
We will first show $\limsup \sigma_n \leq \limsup s_n$. We will follow and show the hint. So we have that 
\begin{align*}
   \sup \left\{\sigma_n: n>M\right\} &= \sup \left\{ \frac{1}{n} (s_1 + \cdots + s_n) : n > M   \right\} \\
   &= \sup \left\{\frac{1}{n} (s_1 + \cdots + s_N) + \frac{1}{n} (s_{N+1} + \cdots + s_n) : n > M \right\} 
\end{align*}

Directly applying the result from problem 3 or Ross 12.4 we get 
\begin{align*}
   \sup \left\{\sigma_n: n>M\right\} & \leq \sup \left\{ \frac{1}{n} (s_1 + \cdots + s_N) : n > M \right\} + \sup \left\{ \frac{1}{n} (s_{N+1} + \cdots + s_n ) : n > M \right\} \\
   & \leq \frac{1}{M} (s_1 + \cdots + s_N) + \sup \left\{ \frac{1}{n} (s_{N+1} + \cdots + s_n ) : n > M \right\}
\end{align*}

Where in the second inequality we use the fact that all $s_i$ in $\sigma_n$ are nonnegative and $n > M$ so we know $\frac{1}{n} \leq \frac{1}{M}$ for all $n > M$. For the second term we have for any $n > M$:
\begin{align*}
\frac{1}{n}(s_{N+1} + \cdots + s_n) &\leq \frac{1}{n}(n-N) \cdot \sup\{s_k: k > N\} \\
&= \frac{n-N}{n} \cdot \sup\{s_k: k > N\} \\
&\leq \sup\{s_k: k > N\}
\end{align*}
Therefore, 
\begin{align*}
   \sup \left\{\sigma_n: n>M\right\} \leq \frac{1}{M} (s_1 + \cdots + s_N) + \sup \{s_n : n > N \}
\end{align*}
Hence, we have $\sup \{ \sigma_n : n > M \} \leq \frac{1}{M} (s_1 + \cdots + s_N) + \sup \{s_n : n > N \}$ for any $ M > N$. Now, we take the limit for fixed $N$ as $M \rightarrow \infty$: 
\begin{align*}
    \limsup \sigma_n &\leq \lim_{M \rightarrow \infty} \frac{1}{M} (s_1 + \cdots + s_N) + \lim_{M \rightarrow \infty} \sup \{ s_N : n > N \} \\
    & = 0 + \sup \{ s_N : n > N \} 
\end{align*}
since $\frac{1}{M}$ goes to $0$ and $\sup \{ s_N : n > N \}$ doesn't depend on $M$. Taking the limit as $N \rightarrow \infty$: 
\begin{align*}
    \lim_{N \rightarrow \infty} \limsup \sigma_n \leq \lim_{N \rightarrow \infty} \sup \{ s_N : n > N \}
\end{align*}
hence,
\begin{align*}
    \limsup \sigma_n \leq \limsup s_n. 
\end{align*}
Note: I could not figure out a clean way to show the other inequality. I will say that $\liminf \sigma_n \leq \limsup \sigma_n$ should hold for any sequence in general (I think we even proved this expliclty on a separate homework), so by transitive property we could conclude that 
\begin{align*}
    \liminf s_n \leq \liminf \sigma_n \leq \limsup \sigma_n \leq \limsup s_n.
\end{align*}
}

\item Show that if $\lim s_n$ exists, then $\lim \sigma_n$ exists and $\lim \sigma_n=$ $\lim s_n$.

\jg{
We are given that $\lim s_n$ exists, let $L = \lim s_n$. By \textbf{Theorem 10.7} this means
\begin{align*}
    \liminf s_n = \lim s_n  = \limsup s_n = L
\end{align*}
From part (a), we have
\begin{align*}
\liminf s_n \leq \liminf \sigma_n \leq \limsup \sigma_n \leq \limsup s_n
\end{align*}

Substituting the above we get
\begin{align*}
L \leq \liminf \sigma_n \leq \limsup \sigma_n \leq L
\end{align*}
Therefore we get
\begin{align*}
L = \liminf \sigma_n = \limsup \sigma_n = L
\end{align*}

Since $\liminf \sigma_n = \limsup \sigma_n$, the limit $\lim \sigma_n$ exists, and
\begin{align*}
\lim \sigma_n = L = \lim s_n
\end{align*}
Therefore, if $\lim s_n$ exists, then $\lim \sigma_n$ also exists and equals $\lim s_n$.

}
\item Give an example where $\lim \sigma_n$ exists, but $\lim s_n$ does not
exist.

\jg{
Consider a sequence $(s_n)$ defined as
\begin{align*}
s_n = 
\begin{cases} 
1 & \text{if } n \text{ is odd} \\
0 & \text{if } n \text{ is even}
\end{cases}
\end{align*}
The sequence simply alternates between $1$ and $0$ so clearly it doesn't converge. We have $\liminf s_n = 0$ and $\limsup s_n = 1$, meaning that $\lim s_n$ does not exist. Now let's consider $\sigma_n = \frac{1}{n} (s_1 + \cdots + s_n)$. For odd $n$ or $2k + 1$ we have 
\begin{align*}
    \sigma_{2k+1} = \frac{1}{2k+1} (s_1 + \cdots + s_{2k+1})
\end{align*}
here, $(s_1 + \cdots + s_{2k+1}) = k + 1$. So, 
\begin{align*}
    \sigma_{2k+1} = \frac{k+1}{2k+1}.
\end{align*}
For even $n = 2k$
\begin{align*}
    \sigma_{2k} &= \frac{1}{2k} (s_1 + \cdots + s_{2k}) \\
    &= \frac{k}{2k} = \frac{1}{2}
\end{align*}
And as $k \rightarrow \infty$ for odd $n$ we have 
\begin{align*}
    \lim_{k \rightarrow \infty} \frac{k+1}{2k+1} = \frac{1}{2}. 
\end{align*}
So therefore, $\lim_{n \rightarrow \infty} \sigma_n = \frac{1}{2}$. Therefore we have given an example of a sequence $(s_n)$ that does not converge i.e. $\lim s_n$ does not exist, but $\lim \sigma_n$ does. 
}
\end{enumerate}
\end{enumerate}

\clearpage
\begin{center}
\vspace*{\fill}
{\Large End of Homework}
\vspace*{\fill}
\end{center}
\end{document}
%%%%%%%%%%%%%%%%%%%%%%%%%%%%%%%%%%%%%%%%%%%%%%%%%%%%%%%%%%%%%%%%%%%%%%

