%%%%%%%%%%%%%%%%%%%%%%%%%%%%%%%%%%%%%
% -*- LaTeX -*- %%%%%%%%%%%%%%%%%%%%%%%%%%%%%%%%%%%%%%%%%%%%%%%%%%%%%%
%
%%%%%%%%%%%%%%%%%%%%%%%%%%%%%%%%%%%%%%%%%%%%%%%%%%%%%%%%%%%%%%%%%%%%%%
%**start of header
\documentclass [10pt]{article}
\usepackage{epsfig}
\usepackage{amsmath,amsfonts,amsthm,amssymb}
\usepackage{setspace}
\usepackage{Tabbing}
\usepackage{fancyhdr}
\usepackage{lastpage}
\usepackage{extramarks}
\usepackage{chngpage}
\usepackage{graphicx,float,wrapfig}
\usepackage{amssymb}
\usepackage{mathtools}
\usepackage{esint}
\usepackage{mathrsfs}
\usepackage{cancel}
\usepackage{xcolor}  
% In case you need to adjust margins:
\topmargin=-0.45in %
\evensidemargin=0in %
\oddsidemargin=0in %
\textwidth=6.5in %
\textheight=9.0in %
\headsep=0.25in %
\newtheorem{theorem}{Theorem}[subsection]
\newtheorem{definition}[theorem]{Definition}
\newtheorem{claim}[theorem]{Claim}
\newtheorem{lemma}[theorem]{Lemma}
\newtheorem{example}[theorem]{Example}
\newtheorem{corollary}[theorem]{Corollary}
\newtheorem{proposition}[theorem]{Proposition}

\newcommand{\jg}[1]{{\color{blue} #1}}

\begin{document}
\begin{center}
{\bf Homework Problems}\\
Math 327, Winter 2025\\
Due 11:30 pm, January 22, 2025
\end{center}

\begin{center}
\jg{
    Jake Gonzales}
\end{center}

{\bf Instructions:} Please include the full problem statement in your submission.
All solutions must be written in legible handwriting
or typed (in each case, the text should be of a reasonable size). Your solutions to
all problems should be written in complete sentences, with proper grammatical
structure.
In all cases where external resources are consulted or used, proper citation must
be given. In addition,
you must provide information on collaboration with your submission: if you worked with others,
or consulted anyone aside from the course staff, in preparing the solutions, their
names should be
listed; if you didn't work with anyone, please indicate this.
If your solutions are not typed, you must scan your written solutions and submit
the digital copy. When submitting problems through LaTex, the LaTex source file
(.tex) must be included in the submission. \\

\jg{
\textbf{Collaborators: N/A }
}

\begin{enumerate}

\item Let $\mathbb{R}[x]$ be the class of single-variable polynomials with real
coefficients. Specifically,
\begin{align*}
\mathbb{R}[x]:= \left\{ a_{m}x^m+ a_{m_1}x^{m-1}+ \cdots +a_1 x + a_0~:~
a_m, a_{m-1}, \dots, a_1, a_0 \in \mathbb{R} \right\}.
\end{align*}
Furthermore, define the class of rational functions, $\mathbb{F}$, by
\begin{align*}
\mathbb{F} := \left\{ \frac{P(x)}{Q(x)}~:~ P, Q \in \mathbb{R}[x], \quad \text{and} \quad  Q \not\equiv 0 \right\}.
\end{align*}
Also, for two elements $f, g \in \mathbb{F}$ define addition and multiplication by
\begin{align*}
(f+g)(x)&:= f(x) +g(x)\\
(f\cdot g)(x) &:= f(x)\cdot g(x).
\end{align*}
We then define the ordering on $\mathbb{F}$ by $f \leq g$ if and only if there
exists $M>0$ such that for all $x>M$
\begin{align*}
f(x) \leq g(x).
\end{align*}
\begin{enumerate}
\item Show that $(\mathbb{F}, +, \cdot, \leq)$ is indeed an ordered field
by verifying the field axioms and the properties of an ordering.

\jg{
For the rest of this question: Let $a,b,c \in \mathbb{F}$ be arbitrary rational functions. By the definition in the problem statement, we can write them as 
\begin{align*}
    a(x) = \frac{P_1(x)}{Q_1(x)}, \quad b(x) = \frac{P_2(x)}{Q_2(x)}, \quad c(x) = \frac{P_3(x)}{Q_3(x)},
\end{align*}
where $P_1, P_2, P_3, Q_1, Q_2, Q_3 \in \mathbb{R}[x]$ and $Q_1, Q_2, Q_3 \neq 0$. \\

\textbf{A1.} $a + (b + c) = (a + b) c$ for all $a,b,c$. 

First, we will compute (b+c) 
\begin{align*}
    (b+c)(x) = b(x) + c(x) = \frac{P_2(x)}{Q_2(x)} + \frac{P_3(x)}{Q_3(x)}.
\end{align*}
To add these two rational functions we will find a common denominator such as 
\begin{align*}
    b(x) + c(x) = \frac{P_2(x) Q_3(x) + P_3(x) Q_2(x)}{Q_2(x) Q_3(x)}.
\end{align*}
Now we add $a$ to $(b + c)$, so then
\begin{align*}
    a + (b+c) = \frac{P_1(x)}{Q_1(x)} + \frac{P_2(x) Q_3(x) + P_3(x) Q_2(x)}{Q_2(x) Q_3(x)}.
\end{align*}
And again we find a common denominator between these 
\begin{align*}
    a+(b+c) = \frac{P_1(x) Q_2(x) Q_3(x) + P_2(x) Q_3(x) Q_1(x) + P_3(x) Q_2(x) Q_1(x)}{ Q_1(x) Q_2(x) Q_3(x)}.
\end{align*}
Now, we compute $(a+b)$ in a similar fashion 
\begin{align*}
    (a+b)(x) = a(x) + b(x) = \frac{P_1(x)}{Q_1(x)} + \frac{P_2(x)}{Q_2(x)} = \frac{P_1(x) Q_2(x) + P_2(x) Q_1(x)}{Q_1(x) Q_2(x)}.
\end{align*}
Adding $c$ to $(a+c)$ 
\begin{align*}
    (a+b) + c 
    &= \frac{P_1(x)}{Q_1(x)} + \frac{P_2(x)}{Q_2(x)} \\
    &= \frac{P_1(x) Q_2(x) + P_2(x) Q_1(x)}{Q_1(x) Q_2(x)} + \frac{P_3(x)}{Q_3(x)} \\
    &= \frac{P_1(x) Q_2(x) Q_3(x) + P_2(x) Q_1(x) Q_3(x) + P_3(x) Q_1(x) Q_2(x)}{Q_1(x) Q_2(x) Q_3(x)}.
\end{align*}
Now by observing both of our answers we see that $a + (b+c) = (a+b) + c$. Thus we have shown $\textbf{A1}$ holds. \\

\textbf{A2.} $a + b = b + a$ for all $a,b$. 

First, we compute $(a+b)$ 
\begin{align*}
    a + b = \frac{P_1(x)}{Q_1(x)} + \frac{P_2(x)}{Q_2(x)} = \frac{P_1(x) Q_2(x) + P_2(x) Q_1(x)}{Q_1(x) Q_2(x)}.
\end{align*}
Then we compute $b + a$ 
\begin{align*}
    b + a = \frac{P_2(x)}{Q_2(x)} + \frac{P_1(x)}{Q_1(x)} = \frac{P_2(x) Q_1(x) + P_1(x) Q_2(x)}{Q_2(x) Q_1(x)}. 
\end{align*}
Now we analyze the numerator and denominator for each result. For the numerator, we observe that $P_1(x) Q_2(x) + P_2(x) Q_1(x)$ is in fact identical to $P_2(x) Q_1(x) + P_1(x) Q_2(x)$ because addition of real numbers is commutative. And for the denominators, $Q_1(x) Q_2(x)$ is identical to $Q_2(x) Q_1(x)$ since multiplication of polynomials is commutative. Therefore, we have shown $\textbf{A2}$ holds. \\

\textbf{A3.} $a + 0 = 0$ for all $a$. 

We first need to define the additive identity in $\mathbb{F}$. It can be defined as the rational function $0(x)$ where $P(x) = 0$ is the zero polynomial where all coefficients are equal to zero and $Q(x) = 1$ is a constant polynomial. Therefore we have $0(x) = P(x) / Q(x) = 0/1 = 0$. Having defined this, let us now compute $(a +0)$ as follows 
\begin{align*}
    (a+0)(x) = a(x) + 0(x) = \frac{P(x)}{Q(x)} + \frac{0}{1}. 
\end{align*}
To add these, we find the common denominator and simplify 
\begin{align*}
    a + 0 = \frac{P(x) \cdot 1 + 0 \cdot Q(x)}{Q(x) \cdot 1} = \frac{P(x)}{Q(x)}. 
\end{align*}
And clearly, when we compare this result to $a = P(x)/Q(x)$, we see that they are identical. Therefore, we have shown \textbf{A3} holds. \\

\textbf{A4.} For each $a$ there is an element $-a$ such that $a + (-a) = 0$.

We first need to define the additive inverse $-a$ in $\mathbb{F}$. We can define it as follows
\begin{align*}
    (-a)(x) = \frac{- P(x)}{Q(x)}
\end{align*}
where $-P(x)$ is the polynomial when every coefficient is multiplied by $-1$. Now, we compute $a + (-a)$ 
\begin{align*}
    (a+ (-a))(x) = a(x) + (-a)(x) = \frac{P(x)}{Q(x)} + \frac{-P(x)}{Q(x)}.
\end{align*}
Adding these together gives
\begin{align*}
    a + (-a) = \frac{P(x) + (-P(x)}{Q(x)}.
\end{align*}
If we simplify the numerator we recover the zero polynomial such that $a + (-a) = 0/Q(x) = 0(x)$ (as shown in the last part). Therefore, we have shown that  
\begin{align*}
    (a+ (-a))(x) = 0(x).
\end{align*}
Thus, we have shown \textbf{A4} holds. \\

\textbf{O1.} Given $a$ and $b$, either $a \leq b$ or $b \leq a$. 

Let's consider the difference between $a(x)$ and $b(x)$ denoted as $d(x)$: 
\begin{align*}
    d(x) = a(x) - b(x) = \frac{P_1(x)}{Q_1(x)} - \frac{P_2(x)}{Q_2(x)} = \frac{P_1(x) Q_2(x) - P_2(x) Q_1(x)}{Q_1(x) Q_2(x)}.
\end{align*}
We know that for large enough $x$, the behavior of the rational function will be determined by the highest degree terms in the numerator and denominator. Therefore, as $x$ grows, the leading coefficient of $d(x)$ will determine whether $d(x)$ is eventually positive or negative. To make this clear, we can write $d(x)$ as 
\begin{align*}
    d(x) \approx \frac{c_N x^n}{c_D x^m} = \frac{c_N}{c_D} x^{n-m}
\end{align*}
where $c_N$ is the leading coefficient for the numerator of $d(x)$ and $c_D$ is the leading coefficient for the denominator of $d(x)$. Then we can check all of the cases as follows
\begin{enumerate}
    \item If $d(x) = a(x) - b(x)$ eventually becomes positive, meaning there exits a point $M$ such that $x>M$ for all $x$, then $a(x) \leq b(x)$. 
    \item If $d(x) = a(x) - b(x)$ eventually becomes negative, meaning there exists a point $M$ such that $x>M$ for all $x$, then $b(x) \leq a(x)$. 
    \item If $h(x)=0$ this means that $a(x) = b(x)$ for all $x$, so $a \leq b$ and $b \leq a$. 
\end{enumerate}
Therefore, we have shown that in all cases, \textbf{O1} holds. \\

\textbf{O2.} If $a\leq b$ and $b\leq a$ then $a=b$. 

We are given that $a \leq b$ and $b \leq a$. From the problem definition we know that this means the following 
\begin{itemize}
    \item $a \leq b$ means there exists $M_1 >0$ such that $a(x) \leq b(x)$ for all $x > M_1$.
    \item $b \leq a$ means there exists $M_2 > 0$ such that $b(x) \leq a(x)$ for all $x > M_2$
\end{itemize}
We are given both are true. Then let $M = \max(M_1, M_2)$. Then we can directly conclude that the above two statements imply $a(x) = b(x)$ for all $x > M$. Thus, we have shown that \textbf{O2} holds. \\

\textbf{O3.} If $a \leq b$ and $b \leq c$, then $a \leq c$. 

We are given that $a \leq b$ and $b \leq c$, and want to show $a \leq c$. From the definition of our problem statement we know that 
\begin{itemize}
    \item $a \leq b$ means there exists $M_1 > 0$ such that for all $x > M_1, a(x) \leq b(x)$. 
    \item $b \leq c$ means there exists $M_2 > 0$ such that for all $x > M_2, b(x) \leq c(x)$.
\end{itemize}
Then we can let $M = \max(M_1, M_2)$ which ensures that for all $x > M$, both $x > M_1$ and $x > M_2$ hold. Therefore, we can conclude that we can combine the results and say there exists an $M$ defined as $M = \max(M_1, M_2)$ such that for all $x > M, a(x) \leq b(x) \leq c(x)$. Therefore, we can conclude that $a(x) \leq c(x)$. Thus, we have shown that \textbf{O3} holds. \\

\textbf{O4.} If $a \leq b$ then $a + c \leq b + c$. 

We are given $a \leq b$ which means there exists $M_1 > 0$ such that for all $x > M$, $a(x) \leq b(x)$. We need to show that $a + c \leq b + c$, which means that we need to find an $M > 0$ such that for all $x > M, (a + c)(x) \leq (b + c)(x)$. 

First we can write out the given inequality that $a \leq b$. Since $a \leq b$, there exits $M_1 > 0$ such that for all $x > M_1$: 
\begin{align*}
    \frac{P_1(x)}{Q_1(x)} \leq \frac{P_2(x)}{Q_2(x)}
\end{align*}
which can be written as $P_1(x) Q_2(x) \leq P_2(x) Q_1(x)$ for all $x > M_1$.

Let's add $c$ to both sides and write out $(a+c)$ and $(b+c)$ as follows 

\begin{align*}
    (a+c)(x) = a(x) + c(x) = \frac{P_1(x)}{Q_1(x)} + \frac{P_3(x)}{Q_3(x)}
\end{align*}
and 
\begin{align*}
    (b+c) = b(x) + c(x) = \frac{P_2(x)}{Q_2(x)} + \frac{P_3(x)}{Q_3(x)}
\end{align*}
Adding these rational functions gives us 
\begin{align*}
    (a+c)(x) = \frac{P_1(x) Q_3(x) + P_3(x) Q_1(x)}{Q_1(x) Q_3(x)}
\end{align*}
and 
\begin{align*}
    (b+c)(x) =  \frac{P_2(x) Q_3(x) + P_3(x) Q_2(x)}{Q_2(x) Q_3(x)}
\end{align*}
If we cross multiply and simplify then this comes down to comparing $P_1(x) Q_2(x)$ vs. $P_2(x) Q_1(x)$, which is expected and by our given inequality we know that for all $x > M_1$, $P_1(x) Q_2(x) \leq P_2(x) Q_1(x)$. Then we can conclude that if $a \leq b$, then for all $x > M$ we have that $(a + c)(x) \leq (b+c)(x)$. Thus, we have shown that \textbf{O4} holds. \\

\textbf{O5.} If $a \leq b$ and $0 \leq c$, then $ac \leq bc$. 

We are given $a\leq b$ which means there exists $M_1 > 0$ such that for all $x > M_1$, $a(x) \leq b(x)$. And that $0 \leq c$ which means there exists $M_2 > 0$ such that for all $x > M_2, 0 \leq c$. 

First, we can write out the inequalities. Since $a \leq b$, there exists $M_1 > 0$ such that for all $x > M_1$ 
\begin{align*}
    \frac{P_1(x)}{Q_1(x)} \leq \frac{P_2(x)}{Q_2(x)}
\end{align*}
which can be written as $P_1(x) Q_2(x) \leq P_2(x) Q_1(x)$ for all $x > M_1$.

Similarly, since $0 \leq c$, there exists $M_2 > 0$ such that for all $x > M_2$ 
\begin{align*}
    0 \leq \frac{P_3(x)}{Q_3(x)}
\end{align*}
which means $0 \leq P_3(x)$ for all $x > M_2$ (since $Q_3(x)$ is always positive by definition). Then if we multiply the inequalities by $c$ we can write out $(ac)$ and $(bc)$ as follows
\begin{align*}
    (ac)(x) = a(x) \cdot c(x) = \frac{P_1(x)}{Q_1(x)} \cdot \frac{P_3(x)}{Q_3(x)}
\end{align*}
and
\begin{align*}
    (bc)(x) = b(x) \cdot c(x) = \frac{P_2(x)}{Q_2(x)} \cdot \frac{P_3(x)}{Q_3(x)}
\end{align*}
Let $M = max(M_1, M_2)$. Then for all $x > M$, we know that both inequalities hold. Since $P_3(x) \geq 0$ for all $x > M$, when we multiply the original inequality $P_1(x) Q_2(x) \leq P_2(x) Q_1(x)$ by $P_3(x)$, the inequality is preserved. Therefore, for all $x > M$ we have that 
\begin{align*}
    P_1(x) Q_2(x) P_3(x) \leq P_2(x) Q_1(x) P_3(x)
\end{align*}
Therefore, we have shown that $(ac)(x) \leq (bc)(x)$ for all $x > M$. Thus, we have shown that \textbf{O5} holds.
}

\item Show that there exists $f$ and $g$ such that for any natural number
$n$ such that $nf \leq g$. Therefore, $(\mathbb{F}, +, \cdot, \leq)$ does not
satisfy the Archimedean Property.

\jg{
From definition 4.6 from the textbook the Archimedean property states that if $a > 0$ and $b > 0$, then for some positive integer $n$, we have $n a > b$. We want to show this does not hold for the above problem statement. 

We can do this by picking arbitrary rational functions for $f$ and $g$ to demonstrate they do not satisfy the Archimedean property. The general idea here is that we can construct a rational function that "grows faster" than the other even if we multiply it by some positive integer $n$. 

Let $f(x) = x$ and $g(x) = x^2$. For any positive integer $n$ we need to show that $n f \leq g$. This means that $n f(x) = nx \leq x^2 = g(x)$ for all $x > M$ for some $M$. For any $n$ we can simply choose $M = n$. Then for all $x > n$ we have $n x < x^2$, since for $x > n, x^2$ grows faster than $nx$. Therefore, we have shown there exists two rational functions $f$ and $g$ that shows $(\mathbb{F}, +, \cdot, \leq)$ does not satisfy the Archimedean property. 
}

\end{enumerate}
\clearpage
%%% Problem 2
\item (Ross 4.7) Let $S$ and $T$ be nonempty bounded subsets of $\mathbb{R}$.

\jg{
First of all, for this problem we know that $\inf S, \inf T, \sup S, \sup T$ exist and are real numbers by the completeness axiom since $S$ and $T$ are nonempty bounded subsets of $\mathbb{R}$. 
}
\begin{enumerate}
\item Prove that if $S \subset T$, then $\inf T \leq \inf S \leq \sup S \leq \sup T$.

\jg{
Suppose $S \subseteq T$. By definition 4.2 and 4.3 from the textbook, we have that $\sup T \geq t$ for all $t \in T$. Since $S \subseteq T$, then clearly $\sup T \geq s$ for all $s \in S$. So, $\sup T$ is greater than (or equal to) the least upper bound for $S$. Which means that $\sup T \geq \sup S$. 

Similarly, $\inf T \leq t$ for all $t \in T$. Since $S \subseteq T$, then clearly $\inf T \leq s$ for all $s \in S$. So $\inf T$ is less than (or equal to) the greatest lower bound for $S$. Which means $\inf T \leq \inf S$. 

Thus, we have proven that if $S \subset T$ then $\inf T \leq \inf S \leq \sup S \leq \sup T$. 
}
\item Prove that $\sup (S \cup T) = \max \{ \sup S, \sup T\}$.

\jg{
We want to show that the supremum of the union of $S$ and $T$ is equal to the $\max$ of $\sup S$ and $\sup T$. 

First, we know that $S$ and $T$ are subsets of the union, so, $S \subseteq S \cup T$ and $T \subseteq S \cup T$. Therefore, by part (a), we know that $\sup S \leq \sup(S \cup T)$ and $\sup T \leq \sup(S \cup T)$. So then, $\max \{ \sup S, \sup T \} \leq \sup(S \cup T)$. 

By definition, $\sup (S\cup T)$ is the least upper bound of $S \cup T$. Therefore, to prove equality, we want to show that $\max \{ \sup S, \sup T\}$ is an upper bound of $S \cup T$. 

We can do this by looking at the elements of $S$ and $T$. Denote $m$ as the elements that are in $S \cup T$. If $m \in S$, then $m \leq \sup S \leq \max \{ \sup S, \sup T\}$. If $m \in T$, then $m \leq \sup T \leq \max \{ \sup S, \sup T\}$. Therefore, $m \leq \max \{ \sup S, \sup T\}$. 

Thus, we have shown that $\sup (S \cup T) = \max \{ \sup S, \sup T\}$.
}
\end{enumerate}
\clearpage
\item (Ross 4.8) Let $S$ and $T$ be nonempty subsets of $\mathbb{R}$ with the
following property: $s \leq t$ for all $s \in S$ and $t \in T$.
\begin{enumerate}
\item Prove that $S$ is bounded above and $T$ is bounded below.

\jg{
We know that $S$ is a nonempty subset of $\mathbb{R}$ and some real number $t$ satisfies $s \leq t$ for all $s \in S$. By definition 4.2 part (a), $t$ is an upper bound of $S$ and the set $S$ is bounded above. 

Similarly, we have that $T$ is a nonempty subset of $\mathbb{R}$ and for some real number $s$ satisfies $s \leq t$ for all $t \in T$, so by definition 4.2 part (b), $s$ is a lower bound of $T$ and the set $T$ is bounded below. 
}

\item Prove that $\sup S \leq \inf T$.

\jg{
We are given $S$ and $T$ as nonempty subsets of $\mathbb{R}$ with the following property: $s \leq t$ for all $s \in S$ and $t \in T$. 

Firstly, by definition 4.3 from the textbook and part (a) of this problem, $S$ is bounded above, so $\sup S$ exists. Similarly, $T$ is bounded below, so $\inf T$ exists. 

We want to show that $\sup S \leq \inf T$. Let us first show that $\sup S$ is a lower bound for $T$. By the given property, for every $s \in S$, we have $s \leq t$, which means that $t$ is an upper bound for $S$. Since $\sup S$ is the least upper bound for $S$, it follows that $\sup S \leq t$. This holds for all $t \in T$, so $\sup S$ is a lower bound for $T$. 

By definition, $\inf T$ is the greatest lower bound of $T$. Therefore, since $\sup S$ is a lower bound of for $T$, and $\inf T$ is the greatest lower bound, we have $\sup S \leq \inf T$. 
}
\item Give an example of such sets $S$ and $T$ where $S \cap T$ is
nonempty.

\jg{
We want to give an example of such sets $S$ and $T$, that presumably satisfy the property $s\leq t$ for all $s \in S$ and $t \in T$, where the intersection of $S$ and $T$ is nonempty. So we just need to find the sets where one element say $m$ is such that $m \in S$ and $m \in T$ and satisfies our properties.  

We can define $S = [0,1]$ and $T = [1,2]$. Clearly, the sets are nonempty. Also, by observation, for all $s \in S$ and $t \in T$ we have that $s \leq t$. And we have a nonempty intersection of $S$ and $T$ (i.e., $S \cap T \neq \emptyset$), since the element $m=1$ is in both $S$ and $T$. 
}
\item Give an example of sets $S$ and $T$ where $\sup S = \inf T$ and $S\cap T= \emptyset.$

\jg{
We want to give an example of sets $S$ and $T$ where the least upper bound of $S$ is equivalent to the greatest lower bound of $T$, and where $S$ and $T$ do not share any elements. 

Let's define the sets as $S = (-\infty, 1)$ and $T = [1, \infty)$. Clearly, both sets are nonempty. Also, by observation, for all $s \in S$ and $t \in T$, we have that $s \leq t$ since $S$ are the set of real numbers less than 1 and $T$ is the set of numbers greater than or equal to 1. 

We also have that $\sup S = 1$ since $1$ is an upper bound for $S$ since all elements in $S$ are strictly less than $1$. Furthermore, $1$ is the least upper bound because any number less than $1$ is in the set $S$. We also have $\inf T = 1$ since $1$ is the smallest element in $T$ and is the greatest lower bound of $T$. Lastly, $S \cap T = \emptyset$ since any number in $S$ is strictly less than $1$ and any number in $T$ is greater or equal to 1, therefore, the sets $S$ and $T$ do not share any elements. 
}
\end{enumerate}
\clearpage
\item (Ross 4.10) Prove that if $a>0$, then there exists $n \in \mathbb{N}$ such
that $\frac{1}{n} < a < n$.

\jg{

Let us first look at the second inequality $a < n$. The Archimedean property in definition 4.6 from the textbook states that if $a > 0$ and $b > 0$ then for some positive integer $n$, we have $na > b$. We will apply to this to our proof where we let $b = a$ and $a=1$. So we have that $1 > 0$ and $a > 0$ then there exists $n_1 \in \mathbb{N}$ such that $n_1 \cdot 1 > a$. Therefore, $n_1 > a$. 

Now, considering the left inequality $1/n > a$. Since $a>0$, we can write the inequality as $1/a > n$. Then, we can directly apply the Archimedean property again. We have $1/a > 0$ and $1 >0$, then there exists $n_2 \in \mathbb{N}$ such that $n_2 \cdot 1 > 1/a$, therefore, $1/n_2 < a$. 

Now let $n = \max (n_1, n_2)$, since $n \geq n_1$, we have $a < n$ and since $n \geq n_2$ we have $1/n > a$. Thus, we have shown that if $a>0$ then there exists $n \in \mathbb{N}$ such that $1/n < a < n$. 
}
\clearpage
\item (Ross 4.14)
Let $A$ and $B$ be nonempty bounded subsets of $\mathbb{R}$, and let $A+B$ be
the set of all sums $a+b$ where $a \in A$ and $b \in B$

\jg{
Since $A$ and $B$ are nonempty bounded subsets of $\mathbb{R}$, then we know that the supremum and infimum exist and are real numbers. 
}
\begin{enumerate}
\item Prove that $\sup (A+B)= \sup A +\sup B.$

\jg{
Let $a \in A$ and $b \in B$. By definition of supremum 
\begin{align*}
    a \leq \sup A \quad \text{and} \quad b \leq \sup B. 
\end{align*}
Adding the inequalities together gives 
\begin{align*}
    a + b \leq \sup A + \sup B.
\end{align*}
This holds for all $a \in A$ and $b \in B$, so $\sup A + \sup B$ is an upper bound for $A + B$. We now want to show that $\sup A + \sup B$ is the least upper bound for $A + B$. 

Let $M$ be any upper bound for $A + B$. We want to show that $\sup A + \sup B \leq M$. For any $a \in A$ lets fix $b \in B$ such that $a + b \leq M$ which is equivalent to $a \leq M - b$. Since this holds for all $a \in A$, $M - b$ is an upper bound for $A$. By the definition of supremum, 
\begin{align*}
    \sup A \leq M - b.
\end{align*}
Now we can rearrange this inequality to give $b \leq M - \sup A$. This holds for all $b \in B$. Therefore $M - \sup A$ is an upper bound on $B$. By definition of supremum, we have
\begin{align*}
    \sup B \leq M - \sup A.
\end{align*}
Rearranging the inequality gives 
\begin{align*}
    \sup A + \sup B \leq M.
\end{align*}
Therefore, we have shown that $\sup A + \sup B$ is an upper bound for $A + B$ and that $\sup A + \sup B$ is the least upper bound for $A + B$. Thus, we have proven that $\sup (A+B) = \sup A + \sup B$. 
}
\item Prove that $\inf(A+B) = \inf A + \inf B.$

\jg{
We will follow a similar proof structure as part (a). Let $a \in A$ and $b \in B$. By definition of infimum 
\begin{align*}
    a \geq \inf A \quad \text{and} \quad b \geq \inf B. 
\end{align*}
Adding the inequalities together gives 
\begin{align*}
    a + b \geq \inf A + \inf B.
\end{align*}
This holds for all $a \in A$ and $b \in B$, so $\inf A + \inf B$ is a lower bound for $A + B$. We now want to show that $\inf A + \inf B$ is the greatest lower bound for $A + B$. 

Let $m$ be any lower bound for $A + B$. We want to show that $\inf A + \inf B \geq m$. For any $a \in A$ lets fix $b \in B$ such that $a + b \geq m$, which is equivalent to $a \geq m - b$. Since this holds for all $a \in A$, $m - b$ is a lower bound for $A$. By definition of infimum, 
\begin{align*}
    \inf A \geq m - b.
\end{align*}
Now we can rearrange this inequality to give $b \geq m - \inf A$. This holds for all $b \in B$. Therefore $m - \inf A$ is a lower bound on $B$. By definition of infimum, we have 
\begin{align*}
    \inf B \geq m - \sup A.
\end{align*}
Rearranging the inequality gives 
\begin{align*}
    \inf A + \inf B \geq m. 
\end{align*}
Therefore, we have shown that $\inf A + \inf B$ is a lower bound for $A + B$ and that $\inf A + \inf B$ is the greatest lower bound for $A + B$. Thus, we have proven that $\inf(A+B) = \inf A + \inf B.$
}
\end{enumerate}
\clearpage
\item (Ross 4.15)
Let $ a, b \in \mathbb{R}.$ Show that if $a \leq b+\frac{1}{n}$ for all $n \in \mathbb{N}$, then $a \leq b.$

\jg{
We are given $a, b \in \mathbb{R}$ such that $a \leq b + 1/n$ for all $n \in \mathbb{N}$. We want to show that $a \leq b$. Suppose, for the sake of contradiction, that $a > b$. Then, $a-b > 0$. By the Archimedean property, there exists some $n \in \mathbb{N}$ such that
\begin{align*}
    n (a-b) > 1.
\end{align*}
Rearranging gives 
\begin{align*}
    (a-b) > \frac{1}{n}.
\end{align*}
Adding $b$ to both sides gives 
\begin{align*}
    a > b + \frac{1}{n}.
\end{align*}
This contradicts our assumption that $a \leq b + 1/n$ for all $n \in \mathbb{N}$. 

Therefore, by contradiction, we have shown that if $a \leq b+\frac{1}{n}$ for all $n \in \mathbb{N}$, then $a \leq b$. 
}
\clearpage
\item (Ross 4.16)
Show that $\sup \{ r \in \mathbb{Q}~:~ r < a\}= a$ for each $a \in \mathbb{R}.$

\jg{
The set $S = \{r \in \mathbb{Q} : r < a\}$ is the set of rational numbers that are less than some real number $a$. The supremum of that set, written $\sup S = \sup \{r \in \mathbb{Q} : r < a\} $, is the least upperbound for that set, so $\sup S \geq r$ for all $r \in S$. To prove that $\sup \{r \in \mathbb{Q} : r < a\} = a$ for each $a \in \mathbb{R}$, by definition 4.2 and 4.3, we need to show that $a$ is an upper bound for the set and that $a$ is the least upper bound of $S$, meaning that no real number smaller than $a$ is an upper bound for $S$. Let's show these two things: 

\textbf{\underline{proof:}} 

By definition $S = \{r \in \mathbb{Q} : r < a\}$, so for every $r \in S$, we have $r < a$. Therefore, $a$ is an upper bound for $S$. 

We now want to show that $a$ is a least upper bound of $S$. To do this, we need to show that for any other real number $b$ such that $b < a$, $b$ is not an upper bound for the set $S$. Meaning, there exists some $r \in S$ such that $r > b$. Suppose $b < a$. Then by Denseness of $\mathbb{Q}$ in definition 4.7 from the textbook, we know that since $a, b \in \mathbb{R}$ and $b < a$ there exists some rational number $r \in \mathbb{Q}$ such that $b < r < a$. This $r$ from the definition satisfies $r \in S$ since $r > a$ and $r < b$. Thus, $b$ cannot be an upper bound for $S$ and therefore $a$ is the least upper bound of $S$. 

Therefore, combining these two steps, we can conclude that $S = \{r \in \mathbb{Q} : r < a\} =a$ for each $a \in \mathbb{R}$. 
}
\end{enumerate}
\clearpage
\begin{center}
\vspace*{\fill}
{\Large End of Homework}
\vspace*{\fill}
\end{center}

\end{document}
%%%%%%%%%%%%%%%%%%%%%%%%%%%%%%%%%%%%%%%%%%%%%%%%%%%%%%%%%%%%%%%%%%%%%%

