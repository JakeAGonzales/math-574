%%%%%%%%%%%%%%%%%%%%%%%%%%%%%%%%%%%%%
% -*- LaTeX -*- %%%%%%%%%%%%%%%%%%%%%%%%%%%%%%%%%%%%%%%%%%%%%%%%%%%%%%
%
%%%%%%%%%%%%%%%%%%%%%%%%%%%%%%%%%%%%%%%%%%%%%%%%%%%%%%%%%%%%%%%%%%%%%%
%**start of header
\documentclass [10pt]{article}
\usepackage{epsfig}
\usepackage{amsmath,amsfonts,amsthm,amssymb}
\usepackage{setspace}
\usepackage{Tabbing}
\usepackage{fancyhdr}
\usepackage{lastpage}
\usepackage{extramarks}
\usepackage{chngpage}
\usepackage{graphicx,float,wrapfig}
\usepackage{xcolor}  
\usepackage{amssymb}
\usepackage{mathtools}
\usepackage{esint}
\usepackage{mathrsfs}
\usepackage{cancel}
% In case you need to adjust margins:
\topmargin=-0.45in %
\evensidemargin=0in %
\oddsidemargin=0in %
\textwidth=6.5in %
\textheight=9.0in %
\headsep=0.25in %
\newtheorem{theorem}{Theorem}[subsection]
\newtheorem{definition}[theorem]{Definition}
\newtheorem{claim}[theorem]{Claim}
\newtheorem{lemma}[theorem]{Lemma}
\newtheorem{example}[theorem]{Example}
\newtheorem{corollary}[theorem]{Corollary}
\newtheorem{proposition}[theorem]{Proposition}

\newcommand{\jg}[1]{{\color{blue} #1}}



\begin{document}
\begin{center}
{\bf Homework Problems}\\
Math 327, Winter 2025\\
Due 11:30 pm, February 26, 2025
\end{center}

\begin{center}
\jg{
    Jake Gonzales}
\end{center}

{\bf Instructions:} Please include the full problem statement in your submission.
All solutions must be written in legible handwriting
or typed (in each case, the text should be of a reasonable size). Your solutions to
all problems should be written in complete sentences, with proper grammatical
structure.
In all cases where external resources are consulted or used, proper citation must
be given. In addition,
you must provide information on collaboration with your submission: if you worked with others,
or consulted anyone aside from the course staff, in preparing the solutions, their
names should be
listed; if you didn't work with anyone, please indicate this.
If your solutions are not typed, you must scan your written solutions and submit
the digital copy. When submitting problems through LaTex, the LaTex source file
(.tex) must be included in the submission.

\jg{
\textbf{Collaborators: N/A }
}

\begin{enumerate}
\item (Ross 12.13)
Let $\left(s_n\right)$ be a bounded sequence in $\mathbb{R}$. Let $A$ be the set of
$a \in \mathbb{R}$ such that $\left\{n \in \mathbb{N}: s_n<a\right\}$ is finite,
i.e., all but finitely many $s_n$ are $\geq a$. Let $B$ be the set of $b \in \mathbb{R}$ such that $\left\{n \in \mathbb{N}: s_n>b\right\}$ is finite. Prove $\sup A=\liminf s_n$ and $\inf B=\limsup s_n$.

\jg{
We are given a set $A$ that consists of all real numbers $a$ such that only finitely  many terms of the sequence $(s_n)$ are less than $a$. And the set $B$ consists of all real numbers $b$ such that only finitely many terms of the sequence $(s_n)$ are greater than $b$. 

We want to prove that $\sup A = \lim \inf s_n$ and $\inf B = \lim \sup s_n$. 

Let's start by proving $\sup A = \lim \inf s_n$. We first want to show that $\lim \inf \leq \sup A$. Let $v_N = \inf \{s_n : n > N \}$ which is the infimum of the sequence $(s_n)$ from $N + 1$ onward where $N \in \mathbb{N}$. Since $v_N$ is the infimum of $\{s_n : n > N \}$, it follows that for $n > N$, we have $s_n \geq v_N$. So the set $\{ n \in \mathbb{N} : s_n < v_n \}$ is in the finite set $\{ 1, \cdots, N \}$. Therefore $v_N \in A$ for all $N$. Since $v_N$ is in $A$, then $v_N \leq \sup A$ for all $N$ by definition. If we take the limit as $N \rightarrow \infty$ then we have 
\begin{align*}
    \lim \inf s_n = \lim_{N \rightarrow \infty} v_N \leq \sup A
\end{align*}
Which shows that $\lim \inf s_n$ is less than or equal to $\sup A$. 

Now we seek to show that $\lim \inf s_N \geq \sup A$. We know that $A$ is the set such that $\{n \in \mathbb{N} : s_n < a \}$ and it is finite. Let $N_a$ be the largest index in the finite set. Then for all $n > N_a$ we have $s_n \geq a$. Now consider again $v_N = \inf \{s_N : n > N\}$. For $N \geq N_a$, since $s_n \geq a$ for all $n > N_a$ it follows that $v_N \geq a$. Then similarly taking the limit as $N \rightarrow \infty$: 
\begin{align*}
    \lim \inf s_n = \lim_{N \rightarrow \infty} v_N \geq a
\end{align*}
And since this holds for all $a \in A$, then $\lim \inf s_n$ is an upper bound for A. Therefore $\lim \inf s_n \geq \sup A$. 

Thus, we can conclude that since $\lim \inf s_n \leq \sup A$ and $\lim \inf s_n \geq \sup A$, this implies $\lim \inf s_n = \sup A$. 

Now let's prove $\inf B = \limsup s_n$. We first want to show that $\limsup s_n \geq \inf B$. Let $u_N = \sup \{s_n : n > N \}$ which is the supremum of the sequence $(s_n)$ from $N + 1$ onward where $N \in \mathbb{N}$. Since $u_N$ is the supremum of $\{s_n : n > N \}$, it follows that for $n > N$, we have $s_n \leq u_N$. So the set $\{ n \in \mathbb{N} : s_n > u_N \}$ is in the finite set $\{ 1, \cdots, N \}$. Therefore $u_N \in B$ for all $N$. Since $u_N$ is in $B$, then $u_N \geq \inf B$ for all $N$ by definition. If we take the limit as $N \rightarrow \infty$ then we have 
\begin{align*}
    \limsup s_n = \lim_{N \rightarrow \infty} u_N \geq \inf B
\end{align*}
Which shows that $\limsup s_n$ is greater than or equal to $\inf B$. 
Now we seek to show that $\limsup s_n \leq \inf B$. We know that $B$ is the set such that $\{n \in \mathbb{N} : s_n > b \}$ and it is finite. Let $N_b$ be the largest index in the finite set. Then for all $n > N_b$ we have $s_n \leq b$. Now consider again $u_N = \sup \{s_n : n > N\}$. For $N \geq N_b$, since $s_n \leq b$ for all $n > N_b$ it follows that $u_N \leq b$. Then similarly taking the limit as $N \rightarrow \infty$: 
\begin{align*}
    \limsup s_n = \lim_{N \rightarrow \infty} u_N \leq b
\end{align*}
And since this holds for all $b \in B$, then $\limsup s_n$ is a lower bound for B. Therefore $\limsup s_n \leq \inf B$. 
Thus, we can conclude that since $\limsup s_n \geq \inf B$ and $\limsup s_n \leq \inf B$, this implies $\limsup s_n = \inf B$.

}

\clearpage
\item (Ross 12.14) Calculate
\begin{enumerate}
\item $\lim_{n \to \infty} (n!)^{1/n}$

\jg{
Let $s_n = n!$. Compute this ratio $\left| \frac{s_{n+1}}{s_n} \right|$ as 
\begin{align*}
    \left| \frac{s_{n+1}}{s_n} \right| = \frac{(n+1)!}{n!} = n+1.
\end{align*}
Observe that as $n \rightarrow \infty$, the above ratio $\rightarrow \infty$. 

By \textbf{Theorem 12.2} from the textbook, we have: 
\begin{align*}
    \lim \sup \left| \frac{s_{n+1}}{s_n} \right| = \infty \implies \lim \sup |s_n|^{1/n} = \infty.
\end{align*}
Since $(n!)^{1/n} = |s_n|^{1/n}$ we can conclude that 
\begin{align*}
    \lim_{n \rightarrow \infty} (n!)^{1/n} = \infty.
\end{align*}
}

\item $\lim_{n \to \infty} \frac{1}{n}(n!)^{1/n}$

\jg{
Let $s_n = \frac{n!}{n^n}$. Then, 
\begin{align*}
    \frac{1}{n} (n!)^{1/n} = \left( \frac{n!}{n^n}\right)^{1/n} = |s_n|^{1/n}. 
\end{align*}
Let's compute the ratio $\left| \frac{s_{n+1}}{s_n} \right|$ as 
\begin{align*}
    \left| \frac{s_{n+1}}{s_n} \right| = \frac{(n+1)!}{(n+1)^{n+1}} \cdot \frac{n^n}{n!} = \frac{n^n}{(n+1)^n} = \left( 1 - \frac{1}{n+1}\right)^n. 
\end{align*}
As $n \rightarrow \infty$, this ratio $\rightarrow e^{-1}$. We know this because a well-used example from the book, $(1+\frac{1}{n})^n$ has been shown to converge to $e$ for large $n$, and this is the inverse. 
By \textbf{Corollary 12.3}, we have 
\begin{align*}
    \lim_{n \rightarrow \infty} \left| \frac{s_{n+1}}{s_n} \right|  = \frac{1}{e} \implies \lim |s_n|^{1/n} = \frac{1}{e}. 
\end{align*}
Thus, $\lim_{n \to \infty} \frac{1}{n}(n!)^{1/n} = \frac{1}{e}$.
}

\end{enumerate}


\clearpage
\item (Ross 14.6)
\begin{enumerate}
\item Prove that if $\sum\left|a_n\right|$ converges and $\left(b_n\right)$ is
a bounded sequence, then $\sum a_n b_n$ converges. Hint: Use Theorem 14.4.

\jg{
First let's recall the Cauchy criterion. A series $\sum c_n$ satisfies the Cauchy criterion if for every $\epsilon > 0$, there exists a number $N$ such that for all $n \geq m > N$, 
\begin{align*}
    \left| \sum_{k=m}^n c_k \right| < \epsilon.
\end{align*}
By \textbf{Theorem 14.4}, a series converges if and only if it satisfies the Cauchy criterion. 

Now, since $\sum\left|a_n\right|$ converges, it satisfies the Cauchy criterion. Therefore, we can say that for every $\epsilon > 0$, there exists a number $N$ such that for al $n \geq m > N$,
\begin{align*}
    \left| \sum_{k=m}^n a_k \right| < \epsilon.
\end{align*}
Since $(b_n)$ is bounded, by definition, there exists an $M > 0$ such that $\left| b_n \right| \leq M$ for all $n$. We want to show that $\sum a_n b_n $ satisfies the Cauchy criterion \emph{i.e.,} for $\epsilon > 0$ there exists an $N$ such that for all $n \geq m > N$
\begin{align*}
    \left| \sum_{k=m}^n a_k b_k \right| < \epsilon.
\end{align*}
Using the triangle inequality and the fact that $(b_n)$ is bounded we get
\begin{align*}
    \left| \sum_{k=m}^n a_k b_k \right| \leq \left| \sum_{k=m}^n a_k b_k \right| \leq M \sum_{k=m}^n \left| a_k \right|. 
\end{align*}
Since $\sum \left| a_n \right|$ satisfies the Cauchy criterion, for some $\epsilon > 0$, there exists $N$ such that $n \geq m > N$,
\begin{align*}
    \sum_{k=m}^n \left| a_k \right| < \frac{\epsilon}{M}.
\end{align*}
Combining the two inequalities, it follows that 
\begin{align*}
    \sum_{k=m}^n \left| a_k \right| < M \cdot \frac{\epsilon}{M} = \epsilon.
\end{align*}
Thus, $\sum a_n b_n$ satisfies the Cauchy criterion. Hence by theorem 14.4, it converges.
}

\item Observe that Corollary 14.7 is a special case of part (a).

\jg{
\textbf{Corollary 14.7} from the textbook states that absolutely convergent series are convergent. This would be a special case of part (a) where the sequence $(b_n)$ is taken to be the constant sequence $b_n = 1$ for all $n$. So in this case $\sum \left|a_n \right |$ converges by definition of absolute convergence. And $(b_n)$ is bounded since its constant. Therefore, by part (a), $\sum a_n b_n = \sum a_n$ converges. 
}

\end{enumerate}


\clearpage
\item Suppose $a_n>0, s_n=a_1+\cdots+a_n$, and $\Sigma a_n$ diverges.
\begin{enumerate}
\item Prove that $\sum \frac{a_n}{1+a_n}$ diverges.

\jg{
We don't get any information on whether $a_n$ is bounded, just that $\sum a_n$ diverges with $a_n > 0$. Therefore, we consider two cases: where $\lim \sup a_n = \infty$ and $a_n$ being bounded. 

\textbf{Case (i):} Assume $\lim \sup a_n = \infty$. Then there exists a subsequence $a_{n_k} \rightarrow \infty$. Then 
\begin{align*}
    \frac{a_{n_k}}{1+a_{n_k}} = \frac{1}{\frac{1}{a_{n_k}} + 1} \rightarrow 1 \quad \text{as} \quad k \rightarrow \infty.
\end{align*}
Since the terms $\frac{a_{n_k}}{1+a_{n_k}}$ do not approach $0$, the series $\sum \frac{a_n}{1+a_n}$  diverges. (This is because for a series $\sum s_n$ to converge, a necessary condition is that $\lim_{n \rightarrow \infty} s_n = 0$). 

\textbf{Case (ii):} Assume $a_n$ is bounded. Therefore, there exists $M>0$ where $a_n \leq M$ for all $n$. Then we can write 
\begin{align*}
    \frac{a_n}{1+a_n} \geq \frac{a_n}{1+M} . 
\end{align*}
Since $\sum \frac{a_n}{1+M}$ diverges (since $\sum a_n$) diverges, by the \textbf{14.2 comparison test (ii)} from the textbook, $\sum \frac{a_n}{1 + a_n}$ also diverges. 

Hence, we've shown that in either case $\sum \frac{a_n}{1+a_n}$ diverges.
}

\item Prove that
$$
\frac{a_{N+1}}{s_{N+1}}+\cdots+\frac{a_{N+k}}{s_{N+k}} \geq 1-\frac{s_N}{s_{N+k}}
$$
and deduce that $\sum \frac{a_n}{s_n}$ diverges.

\jg{
We will prove this by induction. Consider the base case $k=1$, where the left hand side is $\frac{a_{N+1}}{s_{N+1}}$ and the left hand side is $1-\frac{s_N}{s_{N+1}}$. Since $s_{N+1} = s_N + a_{N+1}$ we have 
\begin{align*}
    1 - \frac{s_N}{s_{N+1}} = \frac{a_{N+1}}{s_{N+1}}
\end{align*}
Thus this holds for the base case. Now for the inductive step, we assume this holds for $k$, i.e. 
\begin{align*}
    \sum_{i=1}^k \frac{a_{N+i}}{s_{N+i}} \geq 1 - \frac{s_N}{s_{N+k}}.
\end{align*}
For $k+1$, we add $\frac{a_{N+k+1}}{s_{N+k+1}}$ to both sides to get
\begin{align*}
    \sum_{i=1}^{k+1} \frac{a_{N+1}}{s_{N+i}} \geq 1 - \frac{s_N}{s_{N+k}} + \frac{a_{N+k+1}}{s_{N+k+1}}.  
\end{align*}
Now since $s_{N + k + 1} = s_{N+k} + a_{N+k+1} \geq s_{N+k}$ we have
\begin{align*}
    1 - \frac{s_N}{s_{N+k}} + \frac{a_{N+k+1}}{s_{N+k+1}} \geq 1 - \frac{s_N}{s_{N+k+1}}.
\end{align*}
Thus by mathematical induction, we've proved that $\frac{a_{N+1}}{s_{N+1}}+\cdots+\frac{a_{N+k}}{s_{N+k}} \geq 1-\frac{s_N}{s_{N+k}}$. 

Let us now suppose $\sum a_n$ \emph{converges}. Then by the Cauchy criterion, for $\epsilon = \frac{1}{2}$ there exists $N$ such that for all $k \geq 1$
\begin{align*}
    \sum_{i=N+1}^{N+k} \frac{a_i}{s_i} < \frac{1}{2}. 
\end{align*}
However, since $\sum a_n$ diverges, $s_{N+k} \rightarrow \infty$ as $k \rightarrow \infty$. So choose $k$ large enough so that $s_{N+k} \geq 2_{s_N}$. Then using the inequality above
\begin{align*}
    \sum_{i=N+1}^{N+k} \frac{a_i}{s_i} \geq 1 - \frac{s_N}{s_{N+k}} \geq 1 - \frac{1}{2} = \frac{1}{2}. 
\end{align*}
This contradicts $\epsilon < \frac{1}{2}$. Hence, $\sum \frac{a_n}{s_n}$ diverges.
}

\item Prove that
$$
\frac{a_n}{s_n^2} \leq \frac{1}{s_{n-1}}-\frac{1}{s_n}
$$
and deduce that $\sum \frac{a_n}{s_n^2}$ converges.

\jg{
Since $s_n = s_{n-1} + a_n$ we have 
\begin{align*}
    \frac{1}{s_{n-1}} - \frac{1}{s_n} = \frac{s_n - s_{n-1}}{s_{n-1}s_n}. 
\end{align*}
Observe that $s_n \geq s_{n-1}$, so then $s_{n-1} s_n \leq s_n^2$. Therefore, 
\begin{align*}
    \frac{a_n}{s_n^2} \leq \frac{a_n}{s_{n-1} s_n} = \frac{1}{s_{n-1}} - \frac{1}{s_n}. 
\end{align*}
Now, consider the inequality from $n=2$ to $n=m$: 
\begin{align*}
    \sum_{n=2}^m \frac{a_n}{s_n^2} \leq \sum_{n=2}^m  \left( \frac{1}{s_{n-1}} - \frac{1}{s_n} \right).
\end{align*}
Simplifying the right hand side
\begin{align*}
    \sum_{n=2}^m \frac{a_n}{s_n^2} \leq  \sum_{n=2}^m  \left( \frac{1}{s_{n-1}} - \frac{1}{s_n} \right) = \frac{1}{s_1} - \frac{1}{s_m}
\end{align*}
Since $\sum a_n$ diverges, $s_m \rightarrow \infty$, so $\frac{1}{s_m} \rightarrow 0$. Thus, 
\begin{align*}
    \sum_{n=2}^\infty \frac{a_n}{s_n^2} \leq \frac{1}{s_1}
\end{align*}
and adding the $n=1$ term (some finite value), the series is bounded. Hence, by the comparison test the series converges. 
}



\end{enumerate}


\clearpage
\item Suppose $a_n>0$ and $\Sigma a_n$ converges. Put
$$
r_n=\sum_{m=n}^{\infty} a_m .
$$
\begin{enumerate}
\item Prove that
$$
\frac{a_m}{r_m}+\cdots+\frac{a_n}{r_n}>1-\frac{r_n}{r_m} $$
if $m<n$, and deduce that $\sum \frac{a_n}{r_n}$ diverges.

\jg{
Since we are given $r_k = \sum_{i=k}^\infty a_i$, we can write $r_k = a_k + r_{k+1}$. This implies 
\begin{align*}
    \frac{a_k}{r_k} = 1 - \frac{r_{k+1}}{r_k}. 
\end{align*}
Let's now rewrite the sum that was given as $S = \frac{a_m}{r_m}+\cdots+\frac{a_n}{r_n}>1-\frac{r_n}{r_m}$ and from this we have
\begin{align*}
    S = \left( 1 - \frac{r_{m+1}}{r_m}\right) + \left( 1 - \frac{r_{m+2}}{r_{m+1}}\right) + \cdots + \left( 1 - \frac{r_{n+1}}{r_n} \right). 
\end{align*}
The sum can be further simplified and written as 
\begin{align*}
    S = (n - m + 1) - \left(\frac{r_{m+1}}{r_m} + \frac{r_{m+1}}{r_{m+1}} + \cdots + \frac{r_{n+1}}{r_n}\right)
\end{align*}
Since $r_{k+1} < r_k$ for all $k$, each term $\frac{r_{k+1}}{r_k} < 1$. Therefore,
\begin{align*}
    S > (n - m + 1) - (n - m + 1) \cdot \frac{r_n}{r_m}. 
\end{align*}
Since $(n-m+1) \geq 1$ we have 
\begin{align*}
    S > 1 - \frac{r_n}{r_m}. 
\end{align*}
Thus we have proven $\frac{a_m}{r_m}+\cdots+\frac{a_n}{r_n}>1-\frac{r_n}{r_m} $ if $m < n$. 

We want to show that $\sum \frac{a_n}{r_n}$ diverges. To show this, we will apply the Cauchy criterion. Since $\sum a_n$ converges, $r_n \rightarrow 0$ as $n \rightarrow \infty$. Thus for any fixed $m$, $\frac{r_n}{r_m} \rightarrow 0$ as $n \rightarrow \infty$ . This means that
\begin{align*}
    \sum_{k=m}^n \frac{a_k}{r_k} > 1 - \epsilon
\end{align*}
for large $n$. Thus, the Cauchy criterion is satisfied with $\epsilon = 1$ so $\sum \frac{a_n}{r_n}$ diverges. 

}

\item Prove that
$$
\frac{a_n}{\sqrt{r_n}}<2\left(\sqrt{r_n}-\sqrt{r_{n+1}}\right)
$$
and deduce that $\sum \frac{a_n}{\sqrt{r_n}}$ converges.


\jg{
We can again write $a_n$ in terms of $r$. So since we have $r_n = a_n + r_{n+1}$, we can write $a_n = r_n - r_{n+1}$. If we substitute this into the inequality above: 
\begin{align*}
    \frac{r_n - r_{n+1}}{\sqrt{r_n}} < 2 \left( \sqrt{r_n} - \sqrt{r_{n+1}} \right).
\end{align*}
Dividing both sides by $\sqrt{r_n}$
\begin{align*}
    \frac{r_n - r_{n+1}}{r_n} < 2 \left( 1- \frac{\sqrt{r_{n+1}}}{\sqrt{r_n}} \right)
\end{align*}
The left hand side simplifies to $1- \frac{r_{n+1}}{r_n}$. For notational convenience, let let $x = \frac{\sqrt{r_{n+1}}}{\sqrt{r_n}}$, since $r_{n+1} < r_n$ we have that $x$ must be in between $0 < x < 1$. So let's rewrite the inequality 
\begin{align*}
    1-x^2 < 2(1-x).
\end{align*}
Rearranging gives $x^2 - 2x+1 > 0$ and further $(x-1)^2 > 0$. This is true for all $x = 1$, but since $0 < x < 1$ we can conclude that the inequality holds. 

We will now apply the comparison test. Let's sum both sides of the inequality that we just proved to hold to get: 
\begin{align*}
    \sum_{n=1}^N \frac{a_n}{\sqrt{r_n}} < 2 \sum_{n=1}^N \left(\sqrt{r_n}-\sqrt{r_{n+1}}\right)
\end{align*}
The right hand side is $2(\sqrt{r_1} - \sqrt{r_{N+1}})$. Since $r_{N+1} \rightarrow 0$ as $N \rightarrow \infty$, the right hand side converges to $2 \sqrt{r_1}$. This shows that $\sum \frac{a_n}{\sqrt{r_n}}$ is bounded above by $2 \sqrt{r_1}$. Hence by the comparison test the series converges. 
}


\end{enumerate}
\end{enumerate}

\clearpage
\begin{center}
\vspace*{\fill}
{\Large End of Homework}
\vspace*{\fill}
\end{center}
\end{document}
%%%%%%%%%%%%%%%%%%%%%%%%%%%%%%%%%%%%%%%%%%%%%%%%%%%%%%%%%%%%%%%%%%%%%%


